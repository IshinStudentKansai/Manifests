\documentclass[dvipdfmx]{beamer}
%テーマ
\usetheme[numbering=counter]{corporate}
%パッケージ
\usepackage{xcolor}
%フォント
\renewcommand{\kanjifamilydefault}{\gtdefault}
%色設定
\definecolor{IshinDark}{RGB}{0,85,46}
\definecolor{IshinMain}{RGB}{183,204,71}
\definecolor{IshinBack}{RGB}{38,191,0}
\setbeamercolor*{palette tertiary}{fg=IshinDark, bg=IshinDark}
\setbeamercolor*{palette primary}{fg=IshinBack,bg=IshinBack}
\setbeamercolor*{palette secondary}{fg=IshinMain,bg=IshinMain}

%タイトル設定
\title{2023 東京リージョナルマニフェスト}
\subtitle{自由で豊かな選択肢のあるTokyoへ}
\author[東京維新の会]{東京維新の会}
\date[2023]{2023}

\begin{document}

\maketitle

\begin{frame}{目次}
    \tableofcontents
\end{frame}

\section{教育}
    \begin{frame}[plain,noframenumbering]
        \sectionpage
        \begin{center}
            \begin{large}
                \alert{将来世代へ徹底投資!}\\\alert{新しい時代を生き抜く力を育む教育、無償化への第一歩を}
            \end{large}
        \end{center}
        \begin{small}
            教育無償化へのあゆみを着実に進め、\\
            コロナ禍や物価高で苦しむ学齢期の子ども・子育て世帯の負担を\\
            徹底的に軽減します。\\
            少子化の中で一人ひとりの子どもが持つ可能性を最大限に支援し、\\
            自由で選択肢の多い都市・Tokyoを実現します。
        \end{small}
    \end{frame}
    
    \begin{frame}{給食費無償化+制服等学用品費無償化}
        \begin{small}
            \begin{itemize}
                \setlength{\itemsep}{2mm}
                \item 子どもの心身の健全な育成に欠かせないものとして、コロナ禍でその役割が見直された学校給食を恒久的に無償化し、就学期の子どもを育てる家計を支援します。
                \item 給食無償化に必要な財源について、当面は東京都の助成金と自治体の行財政改革によって捻出する自主財源を組み合わせて確保します。将来的には教育の無償化の一環として、国の責任で実施するよう働きかけていきます。
                \item 学校における私費会計を公会計に移し、隠れ教育費の見える化を行い、制服や標準服、体操服などの高額な学校指定品、その他就学に関する費用の保護者負担を軽減します。また、将来的な義務教育課程において、保護者が負担するこれらの費用の無償化を目指します。
            \end{itemize}
        \end{small}
    \end{frame}

    \begin{frame}{不登校対策}{}
        \begin{small}
            \begin{itemize}
                \setlength{\itemsep}{2mm}
                \item 登校をしないという子どもの選択・意思表示を尊重し、通学校への復帰指導にこだわりすぎることなく、きめ細かく子どもに寄り添うため、フリースクール等の学校外施設とのスムーズな連携を支援します。
                \item 一人一台タブレット等を活用したICT教育を推進し、通学の有無にかかわらず質の高い学びを提供できる体制を構築します。また、不登校児等の当事者の協力も得ながら、VRやメタバースなどを利用した新たな教育手法の開発に取り組みます。
                \item 岐阜市の草潤中学校の「セルフデザイン」や、八王子市の高尾山学園の習熟度別ステップ学習など、全国各地の自治体が先進的な取り組みを行っていることを参考に、子ども達の状況に合わせた新しい教育課程の編成が可能な特例校の設置を推進します。
            \end{itemize}
        \end{small}
    \end{frame}

    \begin{frame}{学校運営への民間ノウハウの徹底活用}{}
        \begin{small}
            \begin{itemize}
                \setlength{\itemsep}{2mm}
                \item 公設民営学校の導入を促進し、競争原理に基づいた人事評価を採用するなど、教育と教員の質の向上を図り、教育環境の変化に柔軟に対応できる体制を構築します。
                \item いわゆる給特法によって教職調整額4\%で教員が過剰勤務を強いられる問題の改善に向け、国の適切な立法措置を促しつつ、当面の間、休日勤務手当や時間外勤務手当については自治体が行財政改革によって捻出した自主財源から支給します。
                \item 学校運営上の事務作業については、民間委託や共同事務センター、教育ICTを徹底的に活用するなど、教員負担の大幅な削減を実現します。
            \end{itemize}
        \end{small}
    \end{frame}

    \begin{frame}{ICTをフル活用した学びの推進}{}
        \begin{small}
            \begin{itemize}
                \setlength{\itemsep}{2mm}
                \item ICTをフル活用した授業を展開し、対面授業とオンライン授業(他校の教員等の動画授業を受けること)を生徒が選択できる仕組みを構築します。
                \item オンライン教育で予習し、教室は学び合い教え合う場とするいわゆる「反転授業」を取り入れ、答えを当てる教育から、答えを見つける教育の導入をリードします。
                \item 生徒と空間を共にする教員は、子どもに教育を施す「先生」ではなく、学び合い教え合う仲間の一人であり、子ども達の可能性を隣で支援する存在であるとの認識の下、教員の役割をアップデートします。
                \item いわゆるブラック校則をなくすため、学校の校則は原則ホームページに公開し、保護者や地域社会の感覚とすり合わせ、児童・生徒の育ちを地域が支援することを推進します。
            \end{itemize}
        \end{small}
    \end{frame}

    \begin{frame}{誰ひとり取り残さない教育}{}
        \begin{small}
            \begin{itemize}
                \setlength{\itemsep}{2mm}
                \item いじめの問題解決を教育現場任せにせず、首長部局にいじめ問題に取り組む組織を設置し、教育現場と役割分担をしながら児童・生徒の命と尊厳を守り、加害者に対しても適切な指導を行います。
                \item 要保護児童対策地域協議会をきめ細かく設置することを通じて、要対協が中心となって地域の社会資源をコーディネートし、教育現場が認知したヤングケアラーの課題解決をはかります。
                \item 困難な状況にある子どもにも学びの機会を保障するため、児童相談所設置の一時保護所から通学ができる環境を整えるとともに、タブレット等学習資材を提供し、活用します。
            \end{itemize}
        \end{small}
    \end{frame}

    \begin{frame}{誰ひとり取り残さない教育}{}
        \begin{small}
            \begin{itemize}
                \setlength{\itemsep}{2mm}
                \item 就学前年度の幼児と保護者が小学校とコミュニケーションを取る機会を増やし、小学校入学後の生活イメージが早期に形成されるよう支援することや、放課後の過ごし方の充実を通じて、いわゆる「小1の壁」の緩和・解消を図ります。
                \item 大阪市が実践している性・生教育を参考に、包括的性教育の取り組みを進め、子ども達一人ひとりの人生設計における選択肢の幅を広げられるように、公教育がその役割を果たせる状況を早期に整備します。
            \end{itemize}
        \end{small}
    \end{frame}

    \begin{frame}{走れる人は後ろを気にせず突進できる教育}{}
        \begin{small}
            \begin{itemize}
                \setlength{\itemsep}{2mm}
                \item 都政とも連携し、 学校教育法や同施行規則に定められている飛び級・飛び入学制度を都立高校に導入可能とする等、特区制度の活用も視野に取り組み、理数系、IT系、技術系の世界的逸材にとって刺激的な環境を整備し、得意分野をさらに伸ばせる仕組みを提供します。
                \item 世界に羽ばたく子ども達の可能性を広げる海外留学について、その支援を大幅に拡充します。また子ども達が海外に飛び出していく意欲を支えるため、オンライン留学を整備します。
                \item 特色ある義務教育学校の設置を支援します。
                \item 子どもたちの商才を引き出し、また困難な状況から脱却する力を身につける機会を提供するため、「子ども起業塾」を実施するなど、マネーリテラシーやビジネスリテラシーを身に着ける教育を推進します。
            \end{itemize}
        \end{small}
    \end{frame}

    \begin{frame}{その他}{}
        \begin{small}
            \begin{itemize}
                \setlength{\itemsep}{2mm}
                \item 地域の協力を得ながら、学校教育の中で郷土の歴史を伝える教育を推進します。
                \item 自殺防止施策を抜本的に見直し、専門家を増やすだけでなく誰もがゲートキーパーになれるよう、相手を否定せず傾聴を意識すること等を教育カリキュラムに取り入れます。
            \end{itemize}
        \end{small}
    \end{frame}

\section{子育て支援}
    \begin{frame}[plain, noframenumbering]
        \sectionpage
        \begin{center}
            \begin{large}
                \alert{出産・育児を罰ゲームにしない!}\\\alert{出産費用無償化と子育て支援の推進}
            \end{large}
        \end{center}
        \begin{small}
            経済的な負担に躊躇することなく子どもを産める制度を整え、\\
            希望する数の子どもを育てられるよう子育て支援を徹底。\\
            「子育て罰」の時代に終止符を打ち、\\
            社会全体で子どもを育む東京を目指します。
        \end{small}
    \end{frame}

    \begin{frame}{子育て支援}{}
        \begin{small}
            \begin{itemize}
                \setlength{\itemsep}{2mm}
                \item 機会平等の社会を作るため、学校給食費のみにとどまらず、教育無償化や最低所得補償制度(ベーシックインカム等)を実現し、家庭の状況によって子どもの可能性が狭まることをなくします。
                \item 出産にかかる医療については、原則保険適用となるよう国に働きかけるとともに、自己負担分相当の「出産育児バウチャー(クーポン)」を支給することで、実質的な出産費用を無償化します。
                \item 子育てに関連するあらゆる助成制度については所得制限を廃止するとともに、ベビーシッター利用料等家事支援経費の事業経費算定を認めるために必要な法改正を促します。
                \item 保育施設入園前の子どもがいる全ての世帯に対して「おむつ定期便」を提供し、訪問型の見守り・相談を定期的に行います。
            \end{itemize}
        \end{small}
    \end{frame}

    \begin{frame}{子どもの命と育ちを守る}{}
        \begin{small}
            \begin{itemize}
                \setlength{\itemsep}{2mm}
                \item 児童相談所の特別区移転を促進し、身近な地域行政として児童相談所の機能を強化します。被虐待児ピアカウンセリング制度を創設するなど、児童相談所が中心的な役割を担った児童福祉の増進をはかります。
                \item 教育・保育施設等での死亡事故は56件、障害の残る事故363件(いずれも2019年)と多発している状況に鑑み、傷害発生時に利用する災害共済給付については病院窓口での負担をなくし、自治体の医療費助成とあわせマイナンバーに統合することで、自己負担・立替払いをなくします。
                \item 要保護児童に家庭的な養育環境を提供するため、里親認定基準を見直し、里親への包括的な支援を抜本強化します。
            \end{itemize}
        \end{small}
    \end{frame}

    \begin{frame}{子どもの命と育ちを守る}{}
        \begin{small}
            \begin{itemize}
                \setlength{\itemsep}{2mm}
                \item ベランダ等の高所からの転落により命を落とす子どもが後を絶たないことから、手すりや柵について規制を見直します。また、重すぎるランドセルやネットいじめなど、子どもの健全な発育を脅かす問題について、迅速に対応していきます。
                \item 教育・保育施設や学校等での水辺での活動におけるライフジャケットの準備、着用の義務化を進めます。
                \item 子どものいる離婚家庭の養育費の取決めを支援するとともに、取り決めた養育費の支払いが滞った時には自治体が一定の立替・督促・回収を行う仕組みを検討します。
                \item 民間事業者と連携し、オンラインを活用した持続可能な24時間365日の医療体制を構築するとともに、医師会・薬剤師会の協力を得て行う休日・夜間診療事業を見直します。
            \end{itemize}
        \end{small}
    \end{frame}

    \begin{frame}{待機児童ゼロの先にある保育の質向上}{}
        \begin{small}
            \begin{itemize}
                \setlength{\itemsep}{2mm}
                \item 委託費の弾力運用に歯止めをかける仕組みを作り、保育士の配置基準上乗せを実施している保育所に十分な委託費の増額を行います。同時に人件費比率についても一定の制限を検討し、人件費の適正化を図ります。
                \item 保育の質の向上のために、園庭のない保育園が利用する公園に代替園庭としての必要な機能を持たせます。
                \item 待機児童が解消している自治体については、新設する保育所には園庭の設置を促す施策を講じます。
                \item 子どもアドボカシーの観点も踏まえ、園内での遊びを含む園運営に5歳児の意見を反映する仕組みを検討します。
            \end{itemize}
        \end{small}
    \end{frame}

    \begin{frame}{いつでも誰でも子どもの預け先に困らない東京へ}{}
        \begin{small}
            \begin{itemize}
                \setlength{\itemsep}{2mm}
                \item 一時保育やファミリーサポートセンターなどの利用登録や予約の利便性向上を図るとともに、ベビーシッター利用時など子どもを預ける際の経済的負担軽減する助成制度を整備します。
                \item 学童クラブ待機児童への対策として、民間学童クラブの活用を推進します。
                \item 待機児童が解消している自治体については、保育の必要性の認定基準の見直しを進めます。とりわけ転職や起業、リスキリングを目的とした就学等についても、就労と同程度の必要性を認定し、保育施設に通う子どもの存在が親の人生の選択肢を狭めている現状を改善します。
            \end{itemize}
        \end{small}
    \end{frame}

    \begin{frame}{男性育休の推進}{}
        \begin{small}
            \begin{itemize}
                \setlength{\itemsep}{2mm}
                \item 男性育休における育児休業給付金が標準月額報酬の80\%となるよう、都と自治体が上乗せ支援を行います。
                \item 第一子出生時における男性育児休業の取得率100\%を目指し、夫婦が家事・育児の共同責任者となることが当たり前の状況を目指し、行政職員の育休取得促進や啓発活動に努めます。
            \end{itemize}
        \end{small}
    \end{frame}

    \begin{frame}{不妊・不育治療の支援}{}
        \begin{small}
            \begin{itemize}
                \setlength{\itemsep}{2mm}
                \item 不妊治療については、保険適用とならない先進医療や最先端の治療等への独自の助成制度を設け、東京が生殖医療の技術開発を牽引していく環境の継続を目指します。また、子どもを産むことに関わるすべての人々が、身体的にも精神的にも社会的にも良好な環境を維持できるよう努めます。
                \item 不妊・不育治療と仕事の両立を支援するために、時間単位の有給休暇の取得を推進します。
            \end{itemize}
        \end{small}
    \end{frame}

\section{行財政改革}
    \begin{frame}[plain, noframenumbering]
        \sectionpage
        \begin{center}
            \begin{large}
                \alert{抜本的な行財政改革を。}\\\alert{シン・東京大改革プランをやり抜く!!}
            \end{large}
        \end{center}
        \begin{small}
            コロナ禍やエネルギー危機、物価高などの影響で\\
            経済的に疲弊している都民に必要な支援を行うため、\\
            今こそ大行政改革を推進します。
        \end{small}
    \end{frame}

    \begin{frame}{デジタルガバメント構想の推進}{}
        \begin{small}
            \begin{itemize}
                \setlength{\itemsep}{2mm}
                \item 申請・届出等の すべての行政手続きをオンライン申請に対応させます。
                \item デジタルサービス局が中心となり、都庁の一部をデジタル空間に仮想的に移転し、都庁第二本庁舎を売却・もしくは民間に貸し出すなど、DXが進む東京都に革新的な取り組みを促します。
                \item 民間の優秀なエンジニア等の採用を強化し、週1日勤務など柔軟な働き方を認めることで、官民間の人材の流動化を促します。
                \item 東京都と都内全自治体をデジタルで常につなげる環境を整え、災害対策等の情報交換をオンラインで瞬時に行える体制を構築します。
            \end{itemize}
        \end{small}
    \end{frame}

    \begin{frame}{デジタルガバメント構想の推進}{}
        \begin{small}
            \begin{itemize}
                \setlength{\itemsep}{2mm}
                \item 自治体の保有するデータについて、徹底した情報公開、オープンデータ化、オープンソース対応を標準化します。
                \item 公開型GISを活用し、街路灯の球切れや道路の補修希望などをスマートフォンから気軽に連絡できるシステムを構築します。
                \item モノづくりのノウハウを活かした農地活用を実現するため、スマート農業の実験的取り組みを行います。
                \item コロナ後の観光産業を立て直すため、その地域に根付いた主体が積極的にプロモーションできる環境を整え、それぞれの地域が持つ観光資源の魅力を最大限引き出すことで地域振興を図ります。
            \end{itemize}
        \end{small}
    \end{frame}

    \begin{frame}{パチンコ、たばこのあり方を抜本的に見直し}{}
        \begin{small}
            \begin{itemize}
                \setlength{\itemsep}{2mm}
                \item 都公安と連携し、パチンコ店の運営や換金方式のあり方を聖域なく見直します。将来的にはパチンコを明確に「公営ギャンブル」と規定して厳格な運用規制と税負担を課し、その財源を地方税としてギャンブル依存症治療等に充てることを目指します。
                \item 自治体が設置している公衆喫煙場所に対し、行政が立ち入り検査を行い、受動喫煙防止の措置が不十分な施設の利用停止処分を行うことができる仕組みを作ります。
                \item たばこ税を目的税と位置づけ、たばこ税収を財源として受動喫煙の起きない密閉型の喫煙場所を整備が促進されるよう、助成を拡充します。
            \end{itemize}
        \end{small}
    \end{frame}

    \begin{frame}{パチンコ、たばこのあり方を抜本的に見直し}{}
        \begin{small}
            \begin{itemize}
                \setlength{\itemsep}{2mm}
                \item たばこ税を目的税化するのと同時に、ふるさと納税による減収部分については、地方交付税交付金不交付団体も減収額の75\%を補填する制度への変更を働きかけます。
                \item たばこ税を活用し、周産期医療の充実や禁煙指導の取組、依存症対策を強化します。
                \item 歩きタバコゼロを目指し、警察と連携した注意喚起を強化するとともに、監視カメラ画像のAI解析による、リアルタイムで機械音声が警告を発する注意喚起の仕組みについて研究を進めます。\par 将来的には歩きタバコの刑事罰化を念頭に、法改正の検討を国に働きかけます。
            \end{itemize}
        \end{small}
    \end{frame}

    \begin{frame}{2030年カーボンハーフに向けて}{}
        \begin{small}
            \begin{itemize}
                \setlength{\itemsep}{2mm}
                \item 2030年カーボンハーフに向けた東京都の取組が進む中、パネルによる太陽光発電のみに捉われることなく、各自治体で最新の技術革新を踏まえた抜本的なエネルギー政策や気候環境問題の議論を進めます。
            \end{itemize}
        \end{small}
    \end{frame}

    \begin{frame}{徹底した議会改革の断行}{}
        \begin{small}
            \begin{itemize}
                \setlength{\itemsep}{2mm}
                \item 公職(政治家)がまず自らを律する姿勢を示します。公正を疑われる金品授受や、議員の不当な口利き、議員の不当な行政人事や外郭団体職員採用への介入を禁止し、罰則を設けます。
                \item 政務活動費の領収書は全てインターネット上で公開します。
                \item 東京都内の地方議会の全公式会議をインターネット(YouTube等)で配信します。
            \end{itemize}
        \end{small}
    \end{frame}
    
\section{感染症対策}
    \begin{frame}[plain, noframenumbering]
        \sectionpage
        \begin{center}
            \begin{large}
                \alert{科学的知見に基づいた感染症対策で、}\\\alert{都民の社会経済活動を復活させる!}
            \end{large}
        \end{center}
        \begin{small}
            政府と東京都は科学的な根拠もなく、\\
            司法でも違法とされた休業要請を乱発し、経済に大打撃を与えました。\\
            曖昧な線引と不正受給が横行した補償制度を抜本的に見直しを行います。\\
            また、医師会とのしがらみに囚われず、\\
            感染拡大期には医療機関への強力な要請と、\\
            平時の規制改革により医療体制を強化します。
        \end{small}
    \end{frame}

    \begin{frame}{コロナ禍を乗り越え、再び活力ある東京へ}{}
        \begin{small}
            \begin{itemize}
                \setlength{\itemsep}{2mm}
                \item コロナ禍の教訓を踏まえ、新興感染症の対応にあたって国、都、基礎自治体それぞれの権限と責任を一致した状態で維持できるよう、予防的体制構築を進めます。
                \item 新興感染症の対策では「自粛を要請」という曖昧な対応をやめ、営業停止を命令として実施し、その際は事業規模にあった十分な補償を必ずセットとします。国政と連携し、マイナンバーと個人口座を紐づけ、命令と同時に補償できる仕組みを構築し、危機を乗り越えるためにデジタルの力を最大限に活用します。
                \item 新興感染症対策や、ワクチン副反応管理等、医療機関に強制力を発揮する際の赤字補填スキームを都内全域に展開し、公立病院・私立病院それぞれが専用病床を迅速に確保できる体制を構築します。
                \item マスクの着用ルールや学校給食における黙食等、国が見直した基準については、デジタル技術を活用し即時に学校等の現場が反映できるよう、行政および学校内の指揮命令系統を整理します。
            \end{itemize}
        \end{small}
    \end{frame}

    \begin{frame}{予防医療の推進で、健康長寿のまち東京へ}{}
        \begin{small}
            \begin{itemize}
                \setlength{\itemsep}{2mm}
                \item 悪化すると人工透析に至るリスクがある、慢性腎臓病の早期診断・早期治療を推進します。
            \end{itemize}
        \end{small}
    \end{frame}

    \begin{frame}{経済振興・生活支援}{}
        \begin{small}
            \begin{itemize}
                \setlength{\itemsep}{2mm}
                \item オリンピック・パラリンピックに向けて整備した、いわゆるレガシー施設を民間に開放し、スポーツ・健康分野のユニコーン企業の輩出を支援します。
                \item コロナ禍で生活が困窮した都民を対象に、毎月定額無利子無担保で借りられる貸付金制度を新設します。その際、マイナンバーと紐付けて所得を捕捉し、所得が一定額に満たない場合は返済を免除(実質給付)する制度を構築します。
            \end{itemize}
        \end{small}
    \end{frame}

\section{減災対策}
    \begin{frame}[plain, noframenumbering]
        \sectionpage
        \begin{center}
            \begin{large}
                \alert{最新の被害想定に基づき、}\\\alert{情勢変化を踏まえた更なる減災対策を!}
            \end{large}
        \end{center}
        \begin{small}
            都心部で急増した高層マンションの自助防災機能を強化し、\\
            電源喪失などの事態を避け在宅避難率を高めます。\\
            また、多摩地区においては一部の公園の地下に大きな空間を作り、\\
            雨水貯留施設や核シェルターとしても利用できるよう環境整備を進めます。
        \end{small}
    \end{frame}

    \begin{frame}{コロナ禍の中での、地震・風水害への対策}{}
        \begin{small}
            \begin{itemize}
                \setlength{\itemsep}{2mm}
                \item 感染拡大局面での災害を想定し、自治体の防災計画や避難所運営を全面的に見直します。一部を除き、都立施設の避難所はその開設・指揮における権限を基礎自治体に移譲し、段ボールベットの備蓄の増加、ペット同行避難対策等きめ細やかな対策を行います。
                \item 被災時の電源環境を守り、また情報喪失を防ぐために、自治体間のスクラム支援や電源・サーバー構成のマルチリージョン化を推進します。
                \item 情報難民を作り出さないため、水や食糧等の支援物資と並んで、充電環境が避難所等で早期に供給できる体制を整備します。
                \item ドローン等を活用した救援物資輸送をシステム化することを通じ、さらなる在宅避難の推奨を図ります。
            \end{itemize}
        \end{small}
    \end{frame}

    \begin{frame}{防災対策に付加価値を}{}
        \begin{small}
            \begin{itemize}
                \setlength{\itemsep}{2mm}
                \item 無電柱化を促進するとともに、交通の要所にIoTセンサーを設けて街の混雑状況を可視化し、リアルタイムの情報共有によって「密」を把握できる仕組みを構築します。
                \item 企業や自治体業務が必要最低限の人員でも回せるよう、事業継続計画の策定をより一層促します。自治体においては、平時の行政スリム化をビルトインした事業継続計画の策定を進め、テレワークができる業務の最大化に努めます。
                \item 都市インフラの補修メンテナンスを進め、防災機能を強化します。またメンテナンスの時期が重なることによる工期遅延、工事費高騰等を避けられるよう、広域的な調整を図って計画的な施設更新を進めます。
                \item 大規模災害に備え、消防操法大会の見直し等、より実践性を重視した訓練及び消防団体制を構築します。
            \end{itemize}
        \end{small}
    \end{frame}

    \begin{frame}{防災対策に付加価値を}{}
        \begin{small}
            \begin{itemize}
                \setlength{\itemsep}{2mm}
                \item 高層建築物の自家発電設備は水没を避けるため、地上部への設置を義務化します。電源喪失のリスクを下げることで、さらなる在宅避難の推進をはかります。
                \item 一定以上の規模の建築物について、一時帰宅困難者の受け入れを義務化し、受け入れに必要な整備に対する助成します。また、新築建築物については容積率の緩和する等の受け入れ促進策を講じます。
                \item 核兵器・生物兵器・化学兵器攻撃等への備えとして、地下空間の活用を促し、有事の際に地域住民に開放可能な避難場所の整備に必要な助成や容積率緩和等を行います。
            \end{itemize}
        \end{small}
    \end{frame}

\section{統治機構改革}
    \begin{frame}[plain, noframenumbering]
        \sectionpage
        \begin{center}
            \begin{large}
                \alert{権限と責任を基礎自治体へ。}\\\alert{民間の底力も徹底活用}
            \end{large}
        \end{center}
        \begin{small}
            特にコロナ禍においては国と自治体の役割分担が曖昧で、\\
            都民は大きな不利益を被りました。\\
            国の権限・財源を都へ、また都の権限・財源は区市町村へ適切に移譲し、\\
            地域の実情を第一に対応できる力強い首都を創ります。\\
            行政にしかできない事業以外は積極的に民間に任せ、\\
            行政機構をスリム化します。
        \end{small}
    \end{frame}

    \begin{frame}{維新はそれでも統治機構改革!}{巨大都市Tokyoの、持続可能な統治機構のあり方を追究}
        \begin{small}
            \begin{itemize}
                \setlength{\itemsep}{2mm}
                \item ニアイズベターの原則に基づき、東京都から特別区に対して大幅な権限・財源の移譲を行うために、平成23年から中断したままとなっている都区のあり方検討委員会を早期に開催します。
                \item 財政調整交付金の配分割合を現在の区55.1%から55%に減らすことを主張している東京都に対しては、児童相談所の区設置などが進む中で「権限は移譲しても財源は譲らない」という得手勝手を許さず、児童虐待防止対策の拡充という政策目的に鑑み、必要な権限と財源をセットで区に移管することを強く求めます。
            \end{itemize}
        \end{small}
    \end{frame}

    \begin{frame}{公共がその役割を果たすため、民間にできることは民間に}{}
        \begin{small}
            \begin{itemize}
                \setlength{\itemsep}{2mm}
                \item 利用者の目線から、現状の都内交通網を最適化してサービスを高めます。具体的には、駅からレンタサイクルを利用する場合や、オフピーク時の移動などにMaaSやダイナミックプライシングの仕組みを導入し、利便性の向上や運賃の低下を促進します。
                \item 公営住宅を民間に売却、または民間への委託を推進します。空き家等の管理運営はURに一元化し、コロナ禍で経済状況が変動し住居確保に困る方々や災害から逃れてきた方等に、一時的に空き家を提供できる仕組みを講じます。
                \item 高い品質と適切な施設管理を行いつつ、経営改革やコスト削減も着実に実践している東京都の水道事業を「輸出」することを目指し、国内及び国外の水道運営に参画できるよう、事業のさらなる民営化促進と積極的なプロモーションを東京都に働きかけます。
            \end{itemize}
        \end{small}
    \end{frame}
    
    \begin{frame}{公共がその役割を果たすため、民間にできることは民間に}{}
        \begin{small}
            \begin{itemize}
                \setlength{\itemsep}{2mm}
                \item 都市公園の活性化のためにPPP、PFIを活用し、民間活力によってより魅力ある公園づくりを目指します。大阪における維新改革の象徴の一つである「てんしば」のように、公園内にカフェなどの利用ニーズが高い施設の設置を促し、その収益で公園の整備等を行うなど、都市公園とその運営の質の向上を図り、個性と賑わいにあふれる公園を増やしていきます。
                \item 公営図書館の役割分担を明確化するとともに、電子書籍中心時代の到来を見据え、自治体運営による図書館のサポート機能を充実させます。また、魅力的な図書館運営の実績のある民間企業の活力導入も検討します。
                \item 自治体が収集する基礎データを自動的にオープンデータ化する仕組みを導入し、国が整備するベース・レジストリと同期する体制を構築します。データの分析や利活用まで行政側だけで行うのではなく、民間の視点や力も活用し、ビジネス機会の提供につなげていくことで、経済成長を後押しします。
            \end{itemize}
        \end{small}
    \end{frame}

\section{多様性社会の実現}
    \begin{frame}[plain, noframenumbering]
        \sectionpage
        \begin{center}
            \begin{large}
                \alert{世界最大の「多様性」と}\\\alert{「表現の自由」都市、東京へ}
            \end{large}
        \end{center}
        \begin{small}
            同性婚や選択的夫婦別姓などの制度を\\
            自治体からも声を上げて後押ししていきます。\\
            「表現の自由」が保障され、それぞれのアイデンティティーや\\
            個性を伸び伸びと表現できる、多様性ある社会を目指します。
        \end{small}
    \end{frame}

    \begin{frame}{多様性社会の実現に不可欠な「表現の自由」を守る}{}
        \begin{small}
            \begin{itemize}
                \setlength{\itemsep}{2mm}
                \item エンターテイメント産業やアーティストに対する支援を行い、表現に対する規制は最小限にとどめます。
                \item パートナーシップ制度の導入にあたっては、性自認を起点とした権利擁護を見直しつつ、社会的障壁を減らす実効性のある制度設計を行います。
                \item 結婚後も旧姓を用いて社会経済活動が行える仕組みの構築を目指し、国に働きかけを強めます。
            \end{itemize}
        \end{small}
    \end{frame}

    \begin{frame}{世界一の文化芸術・エンターテインメント都市の実現}{}
        \begin{small}
            \begin{itemize}
                \setlength{\itemsep}{2mm}
                \item 海外に向けたシティプロモーションを強化し、ロケ地としての活用を促進するなど、フィルムコミッション事業を通じて新たな観光資源を創出します。
                \item コミックマーケット等、世界最大級のエンタメイベントを開催してきたことを都市の資産と位置づけ、感染症対策に十分留意しながら、日本の文化であるコミケ等を後世へ継承し、世界へと発信します。
                \item 不健全図書指定については、表現の自由の見地から安易な指定がなされないよう、非公開での審査方法を改めます。東京都青少年健全育成条例の効果を検証し、過度な制限にならないよう、科学的エビデンスに基づいた内容に見直しをはかります。
            \end{itemize}
        \end{small}
    \end{frame}

    \begin{frame}{多摩地域への移住を促進}{}
        \begin{small}
            \begin{itemize}
                \setlength{\itemsep}{2mm}
                \item 若者のテレワーク環境を整え、ベッドステイタウン(住んで働ける街)として多摩地域のさらなる魅力向上に努めます。
            \end{itemize}
        \end{small}
    \end{frame}

    \begin{frame}{真の動物殺処分ゼロへ}{}
        \begin{small}
            \begin{itemize}
                \setlength{\itemsep}{2mm}
                \item 地方自治体間で連携し、飼い主が不明な猫について避妊去勢手術を手広くサポートできる体制を構築します。飼い主の高齢や健康上に理由により飼育困難になってしまったペットを、次の飼い主へと手渡しできる猫ボランティアを支え、地域での相談支援体制を確立します。
            \end{itemize}
        \end{small}
    \end{frame}

\section{移動手段}
    \begin{frame}[plain, noframenumbering]
        \sectionpage
        \begin{center}
            \begin{large}
                \alert{環境に配慮したアフターコロナの移動手段、}\\\alert{東京をもっとアクティブに楽しむ!}
            \end{large}
        \end{center}
        \begin{small}
            電動キックボードの規制緩和などを通じ、\\
            あたらしいモビリティの導入を進め、\\
            都民一人ひとりの生活スタイルにあった\\
            より快適な移動手段が選べるよう、その選択肢を広げていきます。
        \end{small}
    \end{frame}

    \begin{frame}{ネットワークで繋がる交通、スマートモビリティ}{}
        \begin{small}
            \begin{itemize}
                \setlength{\itemsep}{2mm}
                \item 「移動手段」の確保を暮らしやすさのバロメーターと位置づけ、一人ひとりが自由で自立した移動手段を選択・確保できる豊かな社会を実現することを目指し、自動運転技術の普及を見据えた次世代モビリティ、ライドシェアの導入を促します。
                \item 免許返納後の高齢者や妊産婦、障がい者等の移動を支援するために、新しく多様なモビリティを提供し、生活の質の向上をはかります。
            \end{itemize}
        \end{small}
    \end{frame}

    \begin{frame}{自転車ネットワークを整備}{電車通勤$\rightarrow$ジテツウ}
        \begin{small}
            \begin{itemize}
                \setlength{\itemsep}{2mm}
                \item 満員電車解消のため、都心への高速自転車レーンを整備し、自転車通勤を認めている企業を健康増進と環境配慮の観点で後押しします。
                \item 市場原理を活かしたシェアサイクル業者の支援を検討する等、生活者中心の地域交通の再構築を図ります。
                \item 電動キックボードなどの新たな交通手段については、安全性については当初から厳しい基準を設け、それ以外の規制は極力少ない方法で利用を開始し、事後的に必要な規制を設ける「事後チェックルール」を基本とし、導入を促進します。
                \item 子どもだけでなく、全ての自転車利用者に対するヘルメット着用を促します。
            \end{itemize}
        \end{small}
    \end{frame}

\section{身を切る改革}
    \begin{frame}[plain, noframenumbering]
        \sectionpage
        \begin{center}
            \begin{large}
                \alert{都民に寄り添う、}\\\alert{今こそ「身を切る改革」の実行}
            \end{large}
        \end{center}
        \begin{small}
            コロナ禍や物価高で人々が苦しむ中、\\
            今なお議員だけが過分な報酬を受け取り続けています。\\
            議員報酬・議員定数カットを断行し、\\
            「身を切る改革」により都民に寄り添い、\\
            パンデミック終息と経済復興への覚悟を示します。
        \end{small}
    \end{frame}

    \begin{frame}{東京大改革を断行}{3.11東日本大震災から続けてきた身を切る改革を継続する}
        \begin{small}
            \begin{itemize}
                \setlength{\itemsep}{2mm}
                \item 維新の所属議員は例外なく「身を切る改革」によって覚悟を示し、身分や待遇にこだわらず議会改革・行政改革を断行し、理不尽な税の在り方・使い方を納税者目線で正します。
                \item 役職手当や費用弁償といった過剰な制度は廃止し、交通費は実費支給までとするなど、議員に関する特権的な待遇を抜本的に見直します。
                \item あらゆる既得権に聖域なく切り込み、徹底した 行財政改革で生み出した財源は、次の世代へ徹底投資することを約束します。
            \end{itemize}
        \end{small}
    \end{frame}
    
\end{document}