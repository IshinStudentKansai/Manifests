\documentclass[dvipdfmx]{beamer}
%テーマ
\usetheme[numbering=none]{corporate}
%パッケージ
\usepackage{pxjahyper}
\usepackage{xcolor}
%フォント
\renewcommand{\kanjifamilydefault}{\gtdefault}
%色設定
\definecolor{IshinDark}{RGB}{0,85,46}
\definecolor{IshinMain}{RGB}{183,204,71}
\definecolor{IshinBack}{RGB}{38,191,0}
\setbeamercolor*{palette tertiary}{fg=IshinDark, bg=IshinDark}
\setbeamercolor*{palette primary}{fg=IshinBack,bg=IshinBack}
\setbeamercolor*{palette secondary}{fg=IshinMain,bg=IshinMain}

%タイトル設定
\title{維新が変える、新しい奈良へ}
\subtitle{税金の"使い道"を見直す}
\author{日本維新の会 奈良県総支部}
\date{2023}

\begin{document}

\maketitle

\begin{frame}{目次}
    \tableofcontents
\end{frame}

\begin{frame}{はじめに}{}
    \begin{small}
        \begin{itemize}
            \setlength{\parsep}{.5mm}
            \setlength{\itemsep}{2mm}
            \item 奈良県では、ここ5年間で、コンベンションセンターや奈良公園バスターミナル、なら歴史芸術家村などを初めとした施設の建設に多額の税金が使われました。\par
            これからも老朽化した中央卸売市場の建替え事業、奈良県文化会館の建替え及びその周辺整備費用などが必要となります。
            \item 加えて、近鉄奈良線・大和西大寺駅-近鉄奈良駅間の移設工事に2,000億円、2,000メートルの滑走路を併設した「大規模防災拠点」(五條市)建設に720億円、リニア新駅から関西国際空港までの鉄道建設に1,900億円の巨費が注ぎ込まれようとしていますが、これら途方もない建設費には多額の県民負担が必要となります。\par
            だから、私たちは将来、真に何が必要かを厳しく見極め、税金の使い道を見直して、奈良の暮らしを豊かにします。
        \end{itemize}
    \end{small}
\end{frame}

\section{身を切る改革}
    \begin{frame}[plain, noframenumbering]{}{}
        \sectionpage
        \begin{center}
            \begin{large}
                \alert{政治家、議員の「特権」に}\\\alert{大胆にメスを入れる!}
            \end{large}
        \end{center}
    \end{frame}

    \begin{frame}{身を切る改革}{}
        \begin{small}
            \begin{description}
                \setlength{\parsep}{.5mm}
                \setlength{\itemsep}{2mm}
                \item[知事・市町村長の退職金の廃止!と給与のカット!] \mbox{}\par
                日本維新の会の吉村大阪府知事や松井大阪市長は、公約で退職金を受け取っていません。\par
                奈良県においても維新が「推薦・公認」する知事や市町村長は退職金を受け取らず、給与もカットします。
                \item[議員定数・報酬の削減] \mbox{}\par
                日本維新の会は、大阪府議会の定数を、2011年に109名から88名に、2022年に88名から79名に合計約3割削減しました。\par
                奈良県下の県市町村議会においても、多すぎる議員定数と報酬を削減します。
            \end{description}
        \end{small}
    \end{frame}

\section{徹底した行財政改革}
    \begin{frame}[plain, noframenumbering]{}{}
        \sectionpage
        \begin{center}
            \begin{large}
                \alert{スリムで、}\\\alert{ムダのない行政を実現する!}
            \end{large}
        \end{center}
    \end{frame}

    \begin{frame}{徹底した行財政改革}{}
        \begin{small}
            \begin{description}
                \setlength{\parsep}{.5mm}
                \setlength{\itemsep}{2mm}
                \item[行政のスリム化] \mbox{}\par
                民間にできることは民間に委ね、行政事務のデジタル化を図って行政コストを削減します。
                \item[行政事業レビューの徹底実施] \mbox{}\par
                外部有識者も交えて、事業や補助金を総点検します。税金の使い方にムダはないか、優先順位が正しいかなど、「しがらみゼロ」の日本維新の会がチェックします。
            \end{description}
        \end{small}
    \end{frame}
    
\section{統治機構の改革}
    \begin{frame}[plain, noframenumbering]{}{}
        \sectionpage
        \begin{center}
            \begin{large}
                \alert{「昭和」のままの仕組みを}\\\alert{見直す!}
            \end{large}
        \end{center}
    \end{frame}

    \begin{frame}{統治機構の改革}{}
        \begin{small}
            \begin{description}
                \setlength{\parsep}{.5mm}
                \setlength{\itemsep}{2mm}
                \item[天下り禁止] \mbox{}\par
                奈良県が退職者を関係団体などへ適任者として斡旋する、所謂「天下り」は、ここ3年間で80人にのぼります。\par
                天下りは癒着の温床となり得ることから県や市町村の職員が斡旋で再就職する「特権」の廃止を目指します。
                \item[広域行政の推進] \mbox{}\par
                奈良県は関西広域連合へ全面的に加入し、市町村においては事務処理やゴミ焼却・屎尿処理等の共同化をさらに推進します。
            \end{description}
        \end{small}
    \end{frame}

    \begin{frame}{統治機構の改革}{}
        \begin{small}
            \begin{description}
                \setlength{\parsep}{.5mm}
                \setlength{\itemsep}{2mm}
                \item[監査委員、監査事務局の共同設置] \mbox{}\par
                知事や市町村長が任命した監査委員による監査では、どうしても身内に甘くなり、不正が見過ごされがちです。\par
                県下自治体の共同による監査の一元化を図り、監査機能を強化して不適正な支出がないように税金の使途を厳しく監視するとともに行財政運営の更なる改善や改革を目指します。
            \end{description}
        \end{small}
    \end{frame}
    
\section{次世代への投資}
    \begin{frame}[plain, noframenumbering]{}{}
        \sectionpage
        \begin{center}
            \begin{large}
                \alert{教育・子育て費用負担を}\\\alert{大幅に軽減します!}
            \end{large}
        \end{center}
    \end{frame}

    \begin{frame}{次世代への投資}{}
        \begin{small}
            \begin{description}
                \setlength{\parsep}{.5mm}
                \setlength{\itemsep}{2mm}
                \item[給食費の無償化] \mbox{}\par
                日本維新の会が大阪で実現したように、市町村と県が協力して小学校・中学校の給食費を無償にします。\par
                また、給食のない小中学校に通学する児童・生徒には相当分を助成します。
                \item[塾代、スポーツ教室、習い事への補助] \mbox{}\par
                日本維新の会が大阪で実現したように、頑張る市町村と県が協力して、塾やスポーツ教室、習い事の費用を補助します。(教育バウチャー制度)
                \item[18歳まで医療費無償化] \mbox{}\par
                子どもたちが必要な医療を安心して受けることが出来るように、頑張る市町村と協力して子供たちの医療費を18歳まで無償にします。
            \end{description}
        \end{small}
    \end{frame}

\section{チャレンジを生み出す経済政策}
    \begin{frame}[plain, noframenumbering]{}{}
        \sectionpage
        \begin{center}
            \begin{large}
                \alert{中小・零細企業を}\\\alert{全力で応援します!}
            \end{large}
        \end{center}
    \end{frame}

    \begin{frame}{チャレンジを生み出す経済政策}{}
        \begin{small}
            \begin{description}
                \setlength{\parsep}{.5mm}
                \setlength{\itemsep}{2mm}
                \item[保証人無しの融資制度の創設] \mbox{}\par
                金融機関や信用保証協会と連携して、個人保証無しの融資制度の創設を目指します。
                \item[地元業者への優先発注] \mbox{}\par
                県や市町村が物品を購入するときや工事を発注する場合は、当該自治体内の業者を優先します。\par
                また、工事に要する資材の購入や下請けも、当該自治体内の業者優先を行います。\par
                現状は単なる「お願い」に留まっていることから、地元業者優先の実効性を高めるため、当該自治体内の業者へ発注することができない場合は、その理由を確認する制度を設けます。
            \end{description}
        \end{small}
    \end{frame}

    \begin{frame}{チャレンジを生み出す経済政策}{}
        \begin{small}
            \begin{description}
                \setlength{\parsep}{.5mm}
                \setlength{\itemsep}{2mm}
                \item[土地利用の規制緩和] \mbox{}\par
                県内企業が県外へ流失しないように、現状の厳しすぎる土地利用の規制を見直すとともに企業誘致を推進します。
                \item[スタートアップ(起業)への支援] \mbox{}\par
                県や市町村が管理する基金の運用先として、県や市町村も出資する投資会社(ベンチャーキャピタル)の設立を目指して、奈良でスタートアップ企業を育て、奈良の「働く場所」を増やし、自治体独自収入の増加につなげます。
            \end{description}
        \end{small}
    \end{frame}
    
\section{いきとどいた福祉政策}
    \begin{frame}[plain, noframenumbering]{}{}
        \sectionpage
        \begin{center}
            \begin{large}
                \alert{いつまでも奈良で、}\\\alert{元気に、健やかに!}
            \end{large}
        \end{center}
    \end{frame}

    \begin{frame}{いきとどいた福祉政策}{}
        \begin{small}
            \begin{description}
                \setlength{\parsep}{.5mm}
                \setlength{\itemsep}{2mm}
                \item[がん検診の無償化] \mbox{}\par
                がんも今では早期発見さえすれば治る可能性が高い病気になりました。\par
                県民の生命を守るために、日本人のがんの大半を占める胃がん、大腸がん、肺がん、乳がん、子宮頸がんの検診無償化に向けた取り組みを頑張る市町村と連携して強化します。
                \item[老人ホームの入居費補助] \mbox{}\par
                特別養護老人ホームへの入居待機者に対し、頑張る市町村と県が協力して有料老人ホーム等への入居費用を助成する制度を新設します。
            \end{description}
        \end{small}
    \end{frame}

    \begin{frame}{いきとどいた福祉政策}{}
        \begin{small}
            \begin{description}
                \setlength{\parsep}{.5mm}
                \setlength{\itemsep}{2mm}
                \item[買い物弱者への支援] \mbox{}\par
                過疎や高齢で買い物が困難になっている住民が買い物難民にならないよう、移動販売車の購入を頑張る市町村に補助します。
                \item[垣根のない福祉サービスの推進] \mbox{}\par
                誰もが住み慣れた地域で安心して暮らし続けることができるよう、ニーズに応じた必要なサービスを受けることができる地域拠点づくりを推進します。
            \end{description}
        \end{small}
    \end{frame}

\section{安心、安全の奈良県}
    \begin{frame}[plain, noframenumbering]{}{}
        \sectionpage
        \begin{center}
            \begin{large}
                \alert{「生命」を}\\\alert{守る!}
            \end{large}
        \end{center}
    \end{frame}

    \begin{frame}{安心、安全の奈良県}{}
        \begin{small}
            \begin{description}
                \setlength{\parsep}{.5mm}
                \setlength{\itemsep}{2mm}
                \item[南海トラフ巨大地震や大規模風水害などの自然災害に備える対策の強化] \mbox{}\par
                公的避難施設の拡充や、消防団など地域の防災力を強化します。
                \item[休日夜間救急医療センターの拡充] \mbox{}\par
                救える生命を救うため、休日夜間応急センターの空白エリアを無くし、その医療提供体制を拡充します。
                \item[安心して生み育てることができる産科・小児科医療体制の拡充] \mbox{}\par
                南奈良総合医療センターや、奈良県西和医療センターでさえ分娩できない現状を改善し、お母さんが安心して赤ちゃんを出産することができるよう産科医を増やし、産科・小児科医療体制の拡充を目指します。
            \end{description}
        \end{small}
    \end{frame}

\section{脱炭素社会への推進}
    \begin{frame}[plain, noframenumbering]{}{}
        \sectionpage
        \begin{center}
            \begin{large}
                \alert{温暖化対策、}\\\alert{持続可能な社会の構築に向けて!}
            \end{large}
        \end{center}
    \end{frame}

    \begin{frame}{脱炭素社会への推進}{}
        \begin{small}
            \begin{description}
                \setlength{\parsep}{.5mm}
                \setlength{\itemsep}{2mm}
                \item[電気自動車(EV)の普及支援] \mbox{}\par
                公共施設や、マンションなどの集合住宅においても、EV・PHV用充電設備を普及させるため、市町村と県が協力して、国の充電インフラ補助金に上乗せ支援をします。\par
                また、県の大規模施設にはEV・PHV急速充電設備を追加するとともにEV車の保有台数を増やして防災機能の向上を図ります。
                \item[省エネ家電の買い換え支援] \mbox{}\par
                家庭で使用中の冷蔵庫、エアコン、照明器具などを、省エネ性能の高い製品に買い換える場合、その代金の一部を支援します。\par
                家庭でのCO2排出量を削減するだけでなく、電気代負担の軽減を図ります。
            \end{description}
        \end{small}
    \end{frame}
    
\end{document}