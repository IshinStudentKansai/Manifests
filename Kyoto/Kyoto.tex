\documentclass[dvipdfmx]{beamer}
%テーマ
\usetheme[numbering=counter]{corporate}
%パッケージ
\usepackage{pxjahyper}
\usepackage{xcolor}
%フォント
\renewcommand{\kanjifamilydefault}{\gtdefault}
%色設定
\definecolor{IshinDark}{RGB}{0,85,46}
\definecolor{IshinMain}{RGB}{183,204,71}
\definecolor{IshinBack}{RGB}{38,191,0}
\setbeamercolor*{palette tertiary}{fg=IshinDark, bg=IshinDark}
\setbeamercolor*{palette primary}{fg=IshinBack,bg=IshinBack}
\setbeamercolor*{palette secondary}{fg=IshinMain,bg=IshinMain}

%タイトル設定
\title{京都維新八策}
\subtitle{}
\author{京都維新の会}
\date{2023}

\begin{document}

\maketitle

\begin{frame}{目次}
    \tableofcontents
\end{frame}

\section{新しい京都の政治行政・身を切る改革}
    \begin{frame}[plain, noframenumbering]{}{}
        \sectionpage
        \begin{center}
            \begin{large}
                \alert{身を切る改革で}\\\alert{政治家のあり方を変えていく}
            \end{large}
        \end{center}
    \end{frame}

    \begin{frame}{政治行政・身を切る改革}{}
        \begin{small}
            \begin{itemize}
                \setlength{\itemsep}{2mm}
                \item 京都府議会および市議会における議員定数や議員報酬の削減を断行し、またその特権を廃止することで、政治家の既得権に切り込みます。厳しい行財政改革に挑むため、まずは政治家自らが身を切る改革に取り組みます。
            \end{itemize}
        \end{small}
    \end{frame}

    \begin{frame}{政治行政・身を切る改革}{}
        \begin{small}
            \begin{itemize}
                \setlength{\itemsep}{2mm}
                \item 京都府議会議員の定数を60名から50名に削減し、そのうえ更なる削減を目指します。
                \item 京都府議会議員の議員報酬を3割削減します。
                \item 京都市会議員の定数を67名から60名に削減し、そのうえ更なる削減を目指します。
                \item 京都市会議員の議員報酬を3割削減します。
                \item 府内の市町も同様に、必要に応じた削減を目指します。
                \item 費用弁償を見直します (京都府議会議員1日3000円など) 。
                \item 政務活動費以外の視察費用を見直します(京都市・視察用の別枠を廃止。※京都市議・国内14万円/回人、海外100万円/回・人)。
            \end{itemize}
        \end{small}
    \end{frame}

\section{行財政改革}
    \begin{frame}[plain, noframenumbering]{}{}
        \sectionpage
        \begin{center}
            \begin{large}
                \alert{地方分権と徹底した改革・透明化で}\\\alert{住民主体の政治を目指す}
            \end{large}
        \end{center}
    \end{frame}

    \begin{frame}{行財政改革、公務員改革}{}
        \begin{small}
            \begin{itemize}
                \setlength{\itemsep}{2mm}
                \item 京都維新の会は「京都に住む人たちの手で次の京都を創る」ことを目指しています。常に京都府民の目線に立ち、京都府民が自分たちの住む街の政治に参加しやすい環境を創っていきます。また、京都府、京都市をはじめ多くの地域において必要な行財政の改革に取組みます。その際も民意に基づく政治、実行する政治、透明性のある政治を実行していきます。
            \end{itemize}
        \end{small}
    \end{frame}

    \begin{frame}{行財政改革}{}
        \begin{small}
            \begin{itemize}
                \setlength{\itemsep}{2mm}
                \item 京都府下全自治体の財政の見直し、改革を進めます。京都府では、将来世代に負担を先送りにすることがないよう、収入の範囲内での予算を組み、財政構造を改革していきます。
                \item 京都府下全自治体のおいて、中長期的な視点で、新たな行政経営の仕組みをとりまとめ、歳入の確保、歳出改革に取り組みます。
                \item 京都市において各行政区の裁量を強化して、行政区ごとの特色を活かした街づくりのさらなる実現を目指す。
                \item 昭和から続く補助金や事業を含む、全事業の点検・棚卸を行います。
            \end{itemize}
        \end{small}
    \end{frame}

    \begin{frame}{公務員改革}{}
        \begin{small}
            \begin{itemize}
                \setlength{\itemsep}{2mm}
                \item 公務員を「身分」から「職業」とするため、公務員の過度な身分保障や評価制度を見直し、公務員が生き生きと働ける公務員制度改革を実行します。
                \item 人口減少など新たな課題に直面する社会において、維持すべきは維持しながらも業務の合理化や権限移譲による適切な人員体制の見直しを行い、地方公務員の人員・人件費を削減し、新たな財源を作ります。
                \item 国政とも連携して、人事院勧告制度における官民給与比較のあり方を抜本的に見直し、公務員給与を適正化することで、官民間の実質的な「同一労働同一賃金」を実現します。
                \item 公務員および公務員労働組合による選挙活動を総点検し、特定政党の機関紙購入を含む勤 務時間中の政治活動の禁止を徹底します。
            \end{itemize}
        \end{small}
    \end{frame}

\section{未来への投資・多様性}
    \begin{frame}[plain, noframenumbering]{}{}
        \sectionpage
        \begin{center}
            \begin{large}
                \alert{次世代への徹底投資と}\\\alert{多様性を支える社会の構築}
            \end{large}
        \end{center}
    \end{frame}

    \begin{frame}{未来への投資・多様性}{}
        \begin{small}
            \begin{itemize}
                \setlength{\itemsep}{2mm}
                \item 教育は未来の京都の力です。「こどもまんなか社会」の実現のために、子どもの目線に立った教育や支援を展開していきます。また、多様性を支え、選択肢の多い社会の実現に向けて、京都一丸となって挑戦します。
            \end{itemize}
        \end{small}
    \end{frame}

    \begin{frame}{未来への投資・多様性}{}
        \begin{small}
            \begin{itemize}
                \setlength{\itemsep}{2mm}
                \item 生まれた環境や家庭の経済状況にかかわらず、全ての子どもが等しく質の高い教育を受けることができるよう教育の無償化を目指します。京都府全域で保育の無償化や高校無償化を目指します。
                \item どんな環境に生まれても、子どもたちの可能性を伸ばすために塾や習い事に使えるバウチャー制度を導入します。
                \item 子どもたちが安心して遊べ、災害発生時には避難所になる、住民の憩いの場となる公園の整備を進めていきます。
                \item 学校教育を含む現行の縦割り行政をグレートリセットし、福祉的専門家を学校に配置するなど、貧困やヤングケアラー、さまざまな特性や障がいなど、多様化する子どもたちの課題を解消します。
                \item 保育士不足の解消や保育士の処遇改善など、地域の実情に応じた制度設計を進めていきます。
            \end{itemize}
        \end{small}
    \end{frame}

    \begin{frame}{未来への投資・多様性}{}
        \begin{small}
            \begin{itemize}
                \setlength{\itemsep}{1.5mm}
                \item 特別支援教育では、巡回型通級を増やし、インクルーシブ教育システムを推進します。
                \item 子どもたちに合わせた教育を目指し、一人一台のタブレット端末を使って個別最適な学びや協働的学びを行います。保護者負担となっている高校のタブレット端末については、保護者負担軽減を目指します。
                \item 地域の事情に合わせ、いじめに対応できる体制を構築します。(寝屋川市方式)
                \item 妊娠・出産、育児・介護休業などへの多様なハラスメントへの対策を進めます。働き方改革を推進し、男女が共に仕事と生活を両立しやすい取組を促進します。
                \item 同性婚が認められていない現状を鑑み、京都府下全自治体でのパートナーシップ制度の導入を推進します。(現在は京都市、福知山市、亀岡市、向日市、長岡京市の5市のみ)
            \end{itemize}
        \end{small}
    \end{frame}

\section{規制改革・経済成長戦略}
    \begin{frame}[plain, noframenumbering]{}{}
        \sectionpage
        \begin{center}
            \begin{large}
                \alert{規制改革に正面から挑む、}\\\alert{持続可能で飛躍する京都経済へ}
            \end{large}
        \end{center}
    \end{frame}

    \begin{frame}{規制改革・経済成長戦略}{}
        \begin{small}
            \begin{itemize}
                \setlength{\itemsep}{2mm}
                \item 京都の強みである最先端研究を支援し、また、スタートアップに必要な規制改革を進めます。
                \item 高校を卒業して就職する際の規制を撤廃し、就労の選択肢を増やしていきます。
                \item 担い手不足や高齢化が進む農林漁分野において、若者や障がい者などの就労や生きがいづくりの場を生み出し、産業を始めるために必要な取組や、地域における仲間づくり、新たな働き手の確保と定着に向けた取組を進めます。
            \end{itemize}
        \end{small}
    \end{frame}

    \begin{frame}{規制改革・経済成長戦略}{}
        \begin{small}
            \begin{itemize}
                \setlength{\itemsep}{2mm}
                \item 耕作放棄地の発生は、自然環境の保全に大きな弊害となっています。京都全体の全農地を対象に調査を実施し、遊休・荒廃農地の利用意向を把握するなかで、遊休農地の解消を関係機関と共に目指していきます。
                \item 京都全体で広域的に産学官連携を加速させ、各地域におけるニーズや課題共有を行い、三者一体となり地域社会の変革を目指します。研究開発から事業展開まで柔軟な環境整備をする事で地域課題解決につなげていきます。
                \item 国内外から観光客数・観光消費額の拡大を目指し、関係機関や近隣市町と連携するとともに、豊かな自然や農産物、特色ある歴史・文化を活かした観光戦略を形成し、自立・自走できる特色ある観光地域づくりを推進します。
            \end{itemize}
        \end{small}
    \end{frame}

\section{地域創生・少子化対策}
    \begin{frame}[plain, noframenumbering]{}{}
        \sectionpage
        \begin{center}
            \begin{large}
                \alert{人口流出・超少子高齢化時代の}\\\alert{社会構造に挑む}
            \end{large}
        \end{center}
    \end{frame}

    \begin{frame}{地域創生・少子化対策}{}
        \begin{small}
            \begin{itemize}
                \setlength{\itemsep}{2mm}
                \item 超少子高齢化時代に生きていることを理解し、対策を講じていくことが喫緊の課題です。「京都で働き京都で育てる。」を目標に地域を改革していきます。まずは、未婚化を改善するために、新たな産業のスタートアップを推進し、雇用創出と移住促進、そして婚活事業を組み合わせ強化していきます。また、京都に住む若者が、京都での就職や結婚を希望し続けるための環境整備を行います。デジタルイノベーションを活用し、利便性を向上することで、さまざまな世代のライフスタイルに対応できるデジタル社会を創ります。
            \end{itemize}
        \end{small}
    \end{frame}

    \begin{frame}{地域創生・少子化対策}{}
        \begin{small}
            \begin{itemize}
                \setlength{\itemsep}{2mm}
                \item 妊娠期と出産後で支援が途切れがちになる課題を解決し、家族を包括的に支援する体制構築するため、妊娠期から子育て期に至るまでの切れ目のない支援制度と地域拠点(京都版ネウボラ)を京都全域にて展開します。
                \item 京都府下全域でこども医療費助成を18歳まで拡充します。また、京都版ネウボラとも連携し、赤ちゃんから高齢者までを対象とした地域包括的な医療・見守り体制の構築を目指します。
            \end{itemize}
        \end{small}
    \end{frame}

    \begin{frame}{地域創生・少子化対策}{}
        \begin{small}
            \begin{itemize}
                \setlength{\itemsep}{2mm}
                \item 過疎地域において必要のない規制は積極的に撤廃し、新たな挑戦をしたい企業家が地方を目指す流れをつくります。
                \item 高速インターネットを不自由なく使える環境を整備し、企業の社員がワーケーションを行えるようにし、第一次産業を副業でも営める仕組みを整備することで、交流人口の拡大や地域の活性化を促進します。また、希望する高校生や大学生が一定期間農山漁村にファームステイできる支援制度を創設することで、地方が豊かな国土の保全や食料確保という重要な役割を担うことを若者に啓発するとともに、過疎地の活性化を図ります。
            \end{itemize}
        \end{small}
    \end{frame}

\section{危機管理・地域コミュニティ}
    \begin{frame}[plain, noframenumbering]{}{}
        \sectionpage
        \begin{center}
            \begin{large}
                \alert{危機管理体制と地域コミュニティの再構築で}\\\alert{京都を守る}
            \end{large}
        \end{center}
    \end{frame}
    
    \begin{frame}{危機管理・地域コミュニティ}{}
        \begin{small}
            \begin{itemize}
                \setlength{\itemsep}{2mm}
                \item 一級河川をはじめ多くの河川がある京都。土砂災害も多く発生しています。流域治水の推進、公立学校などの避難所としての機能強化から機動的な救援体制の構築まで、ハードとソフトを組合せた防災・減災を進めます。地域にあった避難所運営を目指し、地域コミュニティの在り方を変えていきます。
            \end{itemize}
        \end{small}
    \end{frame}
    
    \begin{frame}{危機管理・地域コミュニティ}{}
        \begin{small}
            \begin{itemize}
                \setlength{\itemsep}{2mm}
                \item 近所付き合いが希薄になりつつある都市部や、高齢化率の高い地域において、災害発生時に迅速かつ安全な避難や災害弱者へきめ細かい対応ができるよう、地域の防災組織の機能強化や避難計画の策定など、地域コミュニティでの防災機能の強化を促進します。
                \item 避難所のプライバシーや衛生面での環境を改善し、医療関係者などの専門家との連携によるサポート体制を構築した、安らげる避難体制を整備します。
                \item 避難所での授乳スペースや更衣室の確保など、育児や女性に配慮した避難所のあり方を見直します。また災害時には女性や子供に対する犯罪が起こるリスクが高まることから、防犯対策に取り組みます。
            \end{itemize}
        \end{small}
    \end{frame}
    
    \begin{frame}{危機管理・地域コミュニティ}{}
        \begin{small}
            \begin{itemize}
                \setlength{\itemsep}{2mm}
                \item 学校など公的施設の避難所について、災害発生時における運営を円滑に進めるため、備品・備蓄の充実や、通信手段に必要となる非常用電源の確保などインフラ面の強化や、短期避難・長期避難を想定した避難所運営体制を構築します。
                \item ペットを連れて避難ができる環境づくりや制度整備を行います。
                \item 日本に滞在する外国人に考慮し、災害時における行政による情報発信や避難所での多言語対応を充実させます。
                \item 舞鶴市はPAZ(原子力施設から概ね半径5キロ圏内)を有しています。避難計画に沿った避難経路、避難道路の整備や避難を判断する際の情報収集のシステム化など、万が一に備えた危機管理体制を構築します。
            \end{itemize}
        \end{small}
    \end{frame}
    
    \begin{frame}{危機管理・地域コミュニティ}{}
        \begin{small}
            \begin{itemize}
                \setlength{\itemsep}{2mm}
                \item 未整備区間(ミッシングリンク)の解消を含めた道路ネットワークの整備を強力に進め、防災力強化に取り組みます。
                \item 災害時の被害を最小限に抑えるための無電柱化・共同溝建設及び下水管の耐震化を促進します。
                \item インフラの老朽化対策について、ドローンやAI、IoT など最新技術の活用によるメンテナンスの高度化・効率化を継続的に推進するとともに、インフラの集約や不要な施設の撤廃を進めます。
                \item 大規模災害を想定した広域対応や、緊急物資の円滑な輸送体制の構築、防災・減災事業を推進します。
            \end{itemize}
        \end{small}
    \end{frame}
    
    \begin{frame}{危機管理・地域コミュニティ}{}
        \begin{small}
            \begin{itemize}
                \setlength{\itemsep}{2mm}
                \item 災害時の倒木や倒れた電柱の撤去作業において、自治体が所有者を問わず復旧作業できるよう各事業者と協定を結ぶモデル制度(和歌山モデル)を京都府下でも行います。また、中山間地や災害多発地域など公助が及ぶのに時間を要する地域には共助を最大化できる予算を措置し、住民自らによる早期救助、復旧を後押しします。
                \item 土砂災害を誘発する放置人工林を自然林に戻すべく、間伐と広葉樹の植栽を推進します。
                \item 非常用電源のための燃料備蓄をより一層促すとともに、備蓄された燃料の品質劣化に対応するため、適切なチェック体制の整備と燃料入れ替え支援を行い、非常時に停電しない環境を整備します。
            \end{itemize}
        \end{small}
    \end{frame}
    
    \begin{frame}{危機管理・地域コミュニティ}{}
        \begin{small}
            \begin{itemize}
                \setlength{\itemsep}{2mm}
                \item 災害発生時には避難所情報はもとより、水道・電気などのライフラインに関する情報や、学校圏をはじめとした行政施設の運営状況など、市民に必要な多種多様な情報を正確かつ適切なタイミングで発信するため、ICTを効果的に活用した情報発信の仕組みを充実させます。具体的には、SNSを活用したプッシュ型の情報発信や、HPの表示順序の変更など、プル型情報への適切な誘導を行います。
                \item 消防局及び消防団のIT化・DX化について、マイナンバーカードを活用した救急業務の迅速化、救急現場からの情報の取得・デジタル映像の保存と検証の実施、事務処理の作業効率化、デジタル人材の育成などを推進します。
            \end{itemize}
        \end{small}
    \end{frame}

\section{環境保全と次世代エネルギー}
    \begin{frame}[plain, noframenumbering]{}{}
        \sectionpage
        \begin{center}
            \begin{large}
                \alert{持続可能な環境・エネルギー政策で}\\\alert{京都のまちとひとを活性化する}
            \end{large}
        \end{center}
    \end{frame}

    \begin{frame}{環境保全と次世代エネルギー}{}
        \begin{small}
            \begin{itemize}
                \setlength{\itemsep}{2mm}
                \item 京都議定書発祥の地としてカーボンニュートラル実現に向けての取組を強化します。京都全域でのエネルギーの地産地消、省エネなどを通して、持続可能なエネルギー分野での産業化を促進します。また、リサイクルを促進し、ごみ埋め立ての極小化に取り組みます。
            \end{itemize}
        \end{small}
    \end{frame}

    \begin{frame}{環境保全と次世代エネルギー}{}
        \begin{small}
            \begin{itemize}
                \setlength{\itemsep}{2mm}
                \item バイオマス発電、自然エネルギー発電、高効率コジェネレーション発電はもちろん、次世代のトリプルコンバインド発電や石炭ガス化発電など安定・低廉・低 CO2 の新たな電源の開発、誘致、利用を促進します。
                \item HEMS(家庭のエネルギー管理システム)、BEMS(ビルエネルギー管理システム)のさらなる普及、LED、地中熱利用を始めとする省エネルギー技術の積極導入などを通じて世界最先端の省エネルギー都市を目指します。
                \item エネルギー問題を単なる環境問題として環境担当部局が実施するのではなく、未来への産業として位置づけ、担当を産業担当部局へと変更し、新たな投資を呼び込み、技術革新と雇用創出を狙い、産業としての育成を目指します。
            \end{itemize}
        \end{small}
    \end{frame}

    \begin{frame}{環境保全と次世代エネルギー}{}
        \begin{small}
            \begin{itemize}
                \setlength{\itemsep}{2mm}
                \item エネルギーの地産池消を進める為、水素細菌をはじめとするバイオモノづくり分野や、代替エネルギーの研究組織に対して、土地提供や遊休設備貸与、代替エネルギー開発・運営企業への法人税・固定資産税の減免措置などを行い、バイオモノづくりや代替エネルギーの集積地を目指します。
                \item 「豊かな京都の森」を守り育てる為、森林の所有権を明確にする為の条例制定を行い、森林の間伐や林道の整備を実施し、森林再生を促進します。
                \item 琵琶湖から発した水は、京都を経由して大阪湾に注いでいることから、滋賀県・大阪府や関係市町村と連携し、治水、水質・環境保全に一体的に取り組みます。
            \end{itemize}
        \end{small}
    \end{frame}

    \begin{frame}{環境保全と次世代エネルギー}{}
        \begin{small}
            \begin{itemize}
                \setlength{\itemsep}{2mm}
                \item 深刻化する海洋汚染や温暖化の要因とされるプラスチックゴミの削減に向けて、分別や廃棄方法のあり方を適切に見直すなど、処理技術の現状や科学的エビデンスに基づいた対策を進めます。
                \item 府内在住の外国人に対して、大型ゴミも含めて出し方のルール、マナーなど、多様な手法を用い、啓発活動を推進します。
            \end{itemize}
        \end{small}
    \end{frame}

\section{都市外交}
    \begin{frame}[plain, noframenumbering]{}{}
        \sectionpage
        \begin{center}
            \begin{large}
                \alert{京都だからこそ出来る都市外交で、}\\\alert{世界平和に貢献する}
            \end{large}
        \end{center}
    \end{frame}

    \begin{frame}{都市外交}{}
        \begin{small}
            \begin{itemize}
                \setlength{\itemsep}{2mm}
                \item 観光都市京都、自然豊かな京都、文化芸術の街京都。グローバルな課題解決に向けて、議員外交、民間外交ともに活性化していきます。特に、京都の強みである文化発信を使った外交を通じて世界平和に貢献します。
            \end{itemize}
        \end{small}
    \end{frame}

    \begin{frame}{都市外交}{}
        \begin{small}
            \begin{itemize}
                \setlength{\itemsep}{2mm}
                \item 文化庁の京都移転を活用し、歴史的に友好関係諸国との関係を強化し、対話を通じた世界平和の実現に向けて、京都独自の文化外交で役割を果たし貢献します。
                \item いわゆる観光公害など、グローバル規模での課題を世界各国の都市と共有し、問題解決を目指します。
                \item 姉妹都市や世界各国の都市との友好関係を強化するとともに、力による一方的な現状変更の試みは一切容認できない立場を堅持し、国際社会における京都の存在感を活用します。
                \item 大阪・関西万博に対する機運を盛り上げ、京都の魅力を世界に発信していきます。
            \end{itemize}
        \end{small}
    \end{frame}
    
\end{document}