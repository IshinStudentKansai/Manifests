\documentclass[dvipdfmx]{beamer}
%テーマ
\usetheme[numbering=counter]{corporate}
%パッケージ
\usepackage{pxjahyper}
\usepackage{xcolor}
%フォント
\renewcommand{\kanjifamilydefault}{\gtdefault}
%色設定
\definecolor{IshinDark}{RGB}{0,85,46}
\definecolor{IshinMain}{RGB}{183,204,71}
\definecolor{IshinBack}{RGB}{38,191,0}
\setbeamercolor*{palette tertiary}{fg=IshinDark, bg=IshinDark}
\setbeamercolor*{palette primary}{fg=IshinBack,bg=IshinBack}
\setbeamercolor*{palette secondary}{fg=IshinMain,bg=IshinMain}

%タイトル設定
\title{近江八策}
\subtitle{滋賀県の地域特性に応じた独自施策}
\author{日本維新の会 滋賀県総支部}
\date{2023}

\begin{document}

\maketitle

\begin{frame}{目次}
    \tableofcontents
\end{frame}

\section{滋賀から教育改革}
    \begin{frame}[plain, noframenumbering]{}{}
        \sectionpage
    \end{frame}

    \begin{frame}{「県立メタ中、メタ高」を開校}{}
        \begin{small}
            \begin{itemize}
                \setlength{\parsep}{.5mm}
                \setlength{\itemsep}{2mm}
                \item インターネット上につくられた仮想空間「メタバース」、そのメタバース上に県立の中学校・高校を開校します。\par
                現時点では、エンタメ・ゲーム・ファッションといった分野での活用が進んでいますが、県立メタ中メタ高では教育活動全般に活用し、相談事業からいじめ問題・学習支援等様々な専門担当へ繋ぎます。
                \item 大津市で起きた「いじめによる自殺事件」を教訓に早期発見、相談しやすい窓口となり調査権をもつ組織、弁護士をはじめとしてこどもに関わる各分野の専門家たちで構成される「(仮称)こども人権委員会」を県、市町村単位で首長部局に必置します。
            \end{itemize}
        \end{small}
    \end{frame}

\section{滋賀から行政改革}
    \begin{frame}[plain, noframenumbering]{}{}
        \sectionpage
    \end{frame}

    \begin{frame}{合理的かつ効率的な行政を実現する行政改革}{}
        \begin{small}
            \begin{itemize}
                \setlength{\parsep}{.5mm}
                \setlength{\itemsep}{2mm}
                \item 限られた財源の中で政策効果を最大限向上させるためには、政策の遂行に必要な財・サービスの調達を費用対効果において優れたものとすることが不可欠です。\par
                行政のデジタル化を総合的に推進し、情報セキュリティを確保しつつ、ICTの利活用等による業務の省力化・効率化でワンストップ窓口等の利便性向上を図ります。
            \end{itemize}
        \end{small}
    \end{frame}

    \begin{frame}{監査委員制度の機能強化改革}{}
        \begin{small}
            \begin{itemize}
                \setlength{\parsep}{.5mm}
                \setlength{\itemsep}{2mm}
                \item 一部では、首長と友好的な関係者が就いたり、職員の再就職先となったり、議員の名誉職になったりと機能が十分に果たされているとは言い難い状況にあります。\par
                そこで、都道府県の監査には会計検査院のような独立の組織が、これをもって主従関係ができ支配されることがない仕組みを念頭に、市町村に対する監査には、都道府県が派遣する監査委員が監査することとし、配慮しない監査ができ、意見の言える環境をつくります。
            \end{itemize}
        \end{small}
    \end{frame}

\section{滋賀から全国へ水(淡水)環境改革}
    \begin{frame}[plain, noframenumbering]{}{}
        \sectionpage
    \end{frame}
    
    \begin{frame}{滋賀から環境ビジネスを世界に発信}{}
        \begin{small}
            \begin{itemize}
                \setlength{\parsep}{.5mm}
                \setlength{\itemsep}{2mm}
                \item 淡水湖の水環境に特化した研究実績を持つ滋賀だからこそできる事業を推進します。
                \item 淡水湖の水質悪化をはじめあらゆる課題に解決のヒント与えるための研究体制を整え、民間企業と合同で技術を他の都道府県へは勿論、世界の市場に向けて売り込むめるよう推進します。
            \end{itemize}
        \end{small}
    \end{frame}

\section{滋賀から移動交通改革}
    \begin{frame}[plain, noframenumbering]{}{}
        \sectionpage
    \end{frame}
    
    \begin{frame}{滋賀から移動交通改革}{}
        \begin{small}
            \begin{itemize}
                \setlength{\parsep}{.5mm}
                \setlength{\itemsep}{2mm}
                \item EV(電気自動車)やFCV(水素自動車)は、環境にやさしく災害時の電気の確保にも役立ちます。これら自動車産業の拠点地域を目指します。\par
                また、自動運転技術等の最先端技術の研究を進め、慢性的な交通渋滞の緩和を図ります。
                \item 栗東市内の国道1号線にはEV先進地である外国車の営業所が集まっています。\par
                また、EV車の心臓とも言えるリチウム電池を生産する会社が拠点を置かれています。大学など研究機関や県立の工業技術センターもあることから湖南地域をEVやFCV車の研究、製造の拠点とし滋賀県をEV・FCV車に特化した県とします。
            \end{itemize}
        \end{small}
    \end{frame}

\section{滋賀からエネルギー改革}
    \begin{frame}[plain, noframenumbering]{}{}
        \sectionpage
    \end{frame}

    \begin{frame}{エネルギーを自然からの恵みで地産地消}{}
        \begin{small}
            \begin{itemize}
                \setlength{\parsep}{.5mm}
                \setlength{\itemsep}{2mm}
                \item 太陽光や風力等の既存の再生可能エネルギーだけでなく、大中を中心として飼育されている牛から排出される糞尿を使いエネルギーを生み出します。需要の状況次第では水素も作りエネルギーとします。
                \item 他の家畜からの糞尿による発電も可能か検証し、自然の恵み由来のエネルギーを活用できるようにします。
            \end{itemize}
        \end{small}
    \end{frame}

\section{滋賀から介護の改革}
    \begin{frame}[plain, noframenumbering]{}{}
        \sectionpage
    \end{frame}

    \begin{frame}{滋賀から介護の改革}{}
        \begin{small}
            \begin{itemize}
                \setlength{\parsep}{.5mm}
                \setlength{\itemsep}{2mm}
                \item 滋賀県を介護保険制度の特別区とし、介護サービスの単価アップを図ります。介護従事者の給与を上げることにより質の良い介護サービスを県民に提供します。
                \item 介護従事者は、とかく厳しい労働環境にもかかわらず給与額が伴わないことから若者を中心に離職率が高い状況にあります。\par
                国の支援を受け、一律に単価をアップし、質の良いサービスを提供する介護制度を構築する実験を滋賀から実践します。
            \end{itemize}
        \end{small}
    \end{frame}

\section{滋賀から「健幸」を改革}
    \begin{frame}[plain, noframenumbering]{}{}
        \sectionpage
    \end{frame}

    \begin{frame}{滋賀から「健幸」を改革}{}
        \begin{small}
            \begin{itemize}
                \setlength{\parsep}{.5mm}
                \setlength{\itemsep}{2mm}
                \item 元気なまま歳を重ねた事にインセンティブを付与し、健康寿命を延ばします。
                \item 滋賀県民の寿命は日本でトップですが、健康寿命はそれに追いついていません。\par
                病気にならない期間に応じた祝い金の給付や結果として受診が少ない人の医療保険料減額などあらゆる手段を講じ、健康で長生きする県民を増やします。
            \end{itemize}
        \end{small}
    \end{frame}

\section{滋賀から家庭支援のあり方を改革}
    \begin{frame}[plain, noframenumbering]{}{}
        \sectionpage
    \end{frame}

    \begin{frame}{支援を必要とされる家庭への支援を拡充}{}
        \begin{small}
            \begin{itemize}
                \setlength{\parsep}{.5mm}
                \setlength{\itemsep}{2mm}
                \item 教育費等の無償化や多子世帯への支援を通して家庭の教育力を向上させることにより、子どもの学ぶ機会を増やし、確かな学力を身につけることができるようにします。
                \item 支援策として、給食費無償化を中学3年生まで実施\par
                また、医療費の窓口負担無償化を中学3年生まで実施\par
                (仮称)多子累進減免の実施等
            \end{itemize}
        \end{small}
    \end{frame}

\section{近江の秘策 歴史を活かす観光の挑戦}
    \begin{frame}[plain, noframenumbering]{}{}
        \sectionpage
    \end{frame}

    \begin{frame}{「安土城」を再建}{}
        \begin{small}
            \begin{itemize}
                \setlength{\parsep}{.5mm}
                \setlength{\itemsep}{2mm}
                \item 安土城は、織田信長が天下統一に向けての拠点として琵琶湖東岸の安土山に築いた城です。\par
                現在、「幻の安土城」復元プロジェクトとして、デジタル技術を活用した当時の安土城の姿や過去の調査の様子をタブレットやスマートフォン、VR・MRなどにより復元する事業が県事業として進んでいます。
                \item 今日まで幾度も検討されては事業費が課題となり事業化されることがなかった安土城の再建。\par
                経済波及効果の検証はもちろんのこと、徹底的に無駄を省いたうえで補助金・クラウドファンディング等を最大限活用し安土城再建をめざします。
            \end{itemize}
        \end{small}
    \end{frame}
\end{document}