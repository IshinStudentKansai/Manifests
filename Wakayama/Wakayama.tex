\documentclass[dvipdfmx]{beamer}
%テーマ
\usetheme[numbering=counter]{corporate}
%パッケージ
\usepackage{pxjahyper}
\usepackage{xcolor}
%フォント
\renewcommand{\kanjifamilydefault}{\gtdefault}
%色設定
\definecolor{IshinDark}{RGB}{0,85,46}
\definecolor{IshinMain}{RGB}{183,204,71}
\definecolor{IshinBack}{RGB}{38,191,0}
\setbeamercolor*{palette tertiary}{fg=IshinDark, bg=IshinDark}
\setbeamercolor*{palette primary}{fg=IshinBack,bg=IshinBack}
\setbeamercolor*{palette secondary}{fg=IshinMain,bg=IshinMain}

%タイトル設定
\title{維新・和歌山八策}
\subtitle{統治機構の改革から自立した和歌山を創造し未来を切り開く}
\author{日本維新の会 和歌山県総支部}
\date{2023}

\begin{document}

\maketitle

\begin{frame}{目次}
    \tableofcontents
\end{frame}

\section{行財政改革}
    \begin{frame}[plain, noframenumbering]{}{}
        \sectionpage
        \begin{center}
            \begin{large}
                \alert{徹底した行財政改革を遂行し}\\\alert{和歌山の自立を促進する}
            \end{large}
        \end{center}
    \end{frame}

    \begin{frame}{行財政改革}{}
        \begin{small}
            \begin{itemize}
                \setlength{\parsep}{.5mm}
                \setlength{\itemsep}{2mm}
                \item 統治機構改革により中央集権型の国家から地方分権型の国家へ
                \item 地方分権・道州制等を想定しつつ地域と個人が自立出来る社会システムを確立
                \item 地方自らが地域のポテンシャルを最大に活かした個性の光る地方自治を目指す
                \item 住民への最大サービス実現のためのスリムな自治体を創造
                \item 地方の個性をなくす画一的な国主導の公共事業の見直し
            \end{itemize}
        \end{small}
    \end{frame}

    \begin{frame}{行財政改革}{}
        \begin{small}
            \begin{itemize}
                \setlength{\parsep}{.5mm}
                \setlength{\itemsep}{2mm}
                \item 未来の子供たちのため財政規律を徹底して守り、地域財政の健全化を目指す
                \item 大阪方式の行財政改革をベースとして、政治・行政コストを徹底的に削減
                \item 公務員の馴れ合いや様々な慣習を排除して、頑張る人を評価し政策エリートの育成等を目指す
                \item 痛みを伴う改革には、まず政治家自らが身を切りつつ公務員の人件費や不必要な外郭団体等については、聖域の無い徹底した見直しを進める
            \end{itemize}
        \end{small}
    \end{frame}
    
\section{防災・災害対策}
    \begin{frame}[plain, noframenumbering]{}{}
        \sectionpage
        \begin{center}
            \begin{large}
                \alert{自助、共助を全面的にバックアップし}\\\alert{安心・安全を守り抜く防災害対策}
            \end{large}
        \end{center}
    \end{frame}

    \begin{frame}{防災・災害対策}{}
        \begin{small}
            \begin{itemize}
                \setlength{\parsep}{.5mm}
                \setlength{\itemsep}{2mm}
                \item 災害復旧時に投入される公的資金を、地域自らが自由に決定出来る仕組みの構築
                \item 既存の公共施設を更に活用し大地震・津波の避難場所の確保等に利用促進を図る
                \item 高速道路の高機能化(防潮堤としての利用等)による防災対策の推進と補給路の確保に万全を期す
                \item 都市設計として、コンパクトシティ等を視野に入れた居住区域の再配置を検討
            \end{itemize}
        \end{small}
    \end{frame}

    \begin{frame}{防災・災害対策}{}
        \begin{small}
            \begin{itemize}
                \setlength{\parsep}{.5mm}
                \setlength{\itemsep}{2mm}
                \item 災害時には地域の助け合い(共助)が最も重要となり、その為の居住地域における住民間のコミュニケーションを高めていく取組を徹底して進める
                \item 地域自治会と地域でソーシャルサービスを行っているNPO団体等との連携により、日頃から住民の顔が見える地域コミュニケーションの構築を促進する
                \item 災害時には情報の伝達が困難となる。そこでSNSを活用することにより、安否情報等の伝達が行なえるよう活用方法を徹底して検討していく
            \end{itemize}
        \end{small}
    \end{frame}
    
\section{教育・子育て}
    \begin{frame}[plain, noframenumbering]{}{}
        \sectionpage
        \begin{center}
            \begin{large}
                \alert{徹底した教育の重視!}\\\alert{いじめ撲滅への徹底した取組}
            \end{large}
        \end{center}
    \end{frame}

    \begin{frame}{教育・子育て}{}
        \begin{small}
            \begin{itemize}
                \setlength{\parsep}{.5mm}
                \setlength{\itemsep}{2mm}
                \item 和歌山においては全国に比べて子供の読解力や語彙力が非常に低く、その原因と見られる読書量の不足を改善するために、教育現場での読書の推奨をしっかりと進める
                \item 日本文化を尊重し美しい日本語を継承できるように力を入れ、国語力=人間力であるとの考えの下、読み書きを徹底して重視して行っていく
                \item やらされる受け身の勉強から、自らが明確な目的を持って能動的にする勉強へ転換
                \item 維新の会は弱者を徹底的に守る! いじめ撲滅への徹底した取組
                \item インターネットの情報ネットワークを正しく健全に利用できる能力であるネットリテラシーの教育を徹底的に施し、ネット社会にしっかりと適合出来る人材を育て、合わせてネットいじめの撲滅に繋げる
            \end{itemize}
        \end{small}
    \end{frame}
    
\section{雇用}
    \begin{frame}[plain, noframenumbering]{}{}
        \sectionpage
        \begin{center}
            \begin{large}
                \alert{新産業誘致・創造等から}\\\alert{地域の雇用創出}
            \end{large}
        \end{center}
    \end{frame}

    \begin{frame}{雇用}{}
        \begin{small}
            \begin{itemize}
                \setlength{\parsep}{.5mm}
                \setlength{\itemsep}{2mm}
                \item 観光産業の徹底した振興・魅力あるエンターテイメントを集積させ、集客力のある統合型リゾートを生み出し、特に若年層の雇用創出を実現
                \item 植物工場等新産業の創造から高齢者が働きやすい仕組み、雇用体系を作り上げていく
                \item これからの社会においてNPOの存在は、ますます重要となってくる。その中で、NPO団体が活発に活動できる地域社会を目指して税制面の優遇等の支援を行い、NPOへの雇用の機会を大幅に増やす
            \end{itemize}
        \end{small}
    \end{frame}
    
\section{農林水産}
    \begin{frame}[plain, noframenumbering]{}{}
        \sectionpage
        \begin{center}
            \begin{large}
                \alert{これからの時代にマッチした}\\\alert{新しい農林水産業の創造と推進}
            \end{large}
        \end{center}
    \end{frame}

    \begin{frame}{農林水産}{}
        \begin{small}
            \begin{itemize}
                \setlength{\parsep}{.5mm}
                \setlength{\itemsep}{2mm}
                \item 農林水産業への就業やライフスタイルとして、スローライフやロハス等の魅力的なコンセプトを全面に押し出し、Iターン・Uターンを積極的に推し進める
                \item 木質バイオマス発電等の振興により、森林の再整備と林業復興を果たす
                \item 新しい木材建築の振興から紀州木材の需要拡大を図る
                \item 和歌山が日本の水産養殖業の核となれるよう可能性を追求し、クロマグロ養殖の量産拠点となる取組を徹底して進め、養殖クロマグロ=和歌山というブランドの確立を目指す
                \item 農林水産物の開発から流通までを、一貫して行う六次産業等の新しい農林水産業ビジネスの振興を徹底的に図る
            \end{itemize}
        \end{small}
    \end{frame}

\section{観光}
    \begin{frame}[plain, noframenumbering]{}{}
        \sectionpage
        \begin{center}
            \begin{large}
                \alert{着地型観光に力を入れて}\\\alert{民間主体の観光振興策を推進}
            \end{large}
        \end{center}
    \end{frame}

    \begin{frame}{観光}{}
        \begin{small}
            \begin{itemize}
                \setlength{\parsep}{.5mm}
                \setlength{\itemsep}{2mm}
                \item 主要な観光地において観光立県らしい、例えばパークアンドライド方式の導入等の知恵を絞ったアイデアを実現させる
                \item 民間主体で観光事業や観光振興策を推進し、行政は徹底してバックアップ
                \item 地域の特徴を活かした体験型観光の徹底した推進
                \item SNSを活用し日本国内は元より、海外に目を向けた和歌山の着地型観光のプロモーションを推進
            \end{itemize}
        \end{small}
    \end{frame}

\section{道路・交通}
    \begin{frame}[plain, noframenumbering]{}{}
        \sectionpage
        \begin{center}
            \begin{large}
                \alert{和歌山内における渋滞の緩和と}\\\alert{車のスムーズな運行}
            \end{large}
        \end{center}
    \end{frame}

    \begin{frame}{観光}{}
        \begin{small}
            \begin{itemize}
                \setlength{\parsep}{.5mm}
                \setlength{\itemsep}{2mm}
                \item 住民の要望にしっかりと応える道路政策への取組として、未改良道路の早期改良化や慢性的な渋滞の発生場所、通行が困難な個所へのバイパス敷設、拡幅整備主要駅等へのパークアンドライド方式導入による公共交通機関の利用の促進
                \item 和歌山~海南間の渋滞緩和(それを補う他の幹線道路の整備敷設)
                \item 第二阪和道路の早期完成と海南地方への延伸化
                \item 府県道路の整備促進
                \item 京奈和自動車道の早期完成
            \end{itemize}
        \end{small}
    \end{frame}

    \begin{frame}{観光}{}
        \begin{small}
            \begin{itemize}
                \setlength{\parsep}{.5mm}
                \setlength{\itemsep}{2mm}
                \item 国道42号線海南有田バイパスの早期実現
                \item 国道370号線と371号線の整備促進
                \item 国道424号線の整備促進
                \item 粉河加太線と近畿自動車道有田~田辺間の4車線化
                \item 高速道路の紀伊半島一周の早期実現
                \item サービスエリアや道の駅等の活性化
            \end{itemize}
        \end{small}
    \end{frame}

\section{新産業創造}
    \begin{frame}[plain, noframenumbering]{}{}
        \sectionpage
        \begin{center}
            \begin{large}
                \alert{地域の特徴を活かした}\\\alert{観光医療産業の実現}
            \end{large}
        \end{center}
    \end{frame}

    \begin{frame}{新産業創造}{}
        \begin{small}
            \begin{itemize}
                \setlength{\parsep}{.5mm}
                \setlength{\itemsep}{2mm}
                \item 従来の観光と予防医療・ヘルスケアを融合させた観光医療産業を創造し、徹底して推進する(医療特区も視野に入れて
                \item 「茶がゆ」等の和歌山特産の健康志向の強い商品をさらに開発していく
                \item 医療・ヘルスケアという視点から、和歌山の特徴でもある温泉等の自然資産の活用
                \item 和歌山の素晴らしい海を活用して、イルカセラピーやタラソテラピーを実践
                \item ホテルや旅館への観光医療・ヘルスケアサービスの導入を、全面的にバックアップしていく補助制度等を整備し観光医療産業の振興に繋げる
            \end{itemize}
        \end{small}
    \end{frame}
    
\end{document}