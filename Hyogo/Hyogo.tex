\documentclass[dvipdfmx]{beamer}
%テーマ
\usetheme[numbering=counter]{corporate}
%パッケージ
\usepackage{pxjahyper}
\usepackage{xcolor}
%フォント
\renewcommand{\kanjifamilydefault}{\gtdefault}
%色設定
\definecolor{IshinDark}{RGB}{0,85,46}
\definecolor{IshinMain}{RGB}{183,204,71}
\definecolor{IshinBack}{RGB}{38,191,0}
\setbeamercolor*{palette tertiary}{fg=IshinDark, bg=IshinDark}
\setbeamercolor*{palette primary}{fg=IshinBack,bg=IshinBack}
\setbeamercolor*{palette secondary}{fg=IshinMain,bg=IshinMain}

%タイトル設定
\title{兵庫維新八策}
\subtitle{兵庫の成長力を最大化}
\author{兵庫維新の会}
\date{2023}

\begin{document}

\maketitle

\begin{frame}{目次}
    \tableofcontents
\end{frame}

\section{行財政改革}
    \begin{frame}[plain, noframenumbering]{}{}
        \sectionpage
        \begin{center}
            \begin{large}
                \alert{さらなる行財政改革の断行と、}\\\alert{デジタル社会に対応した新しい行政の実現}
            \end{large}
        \end{center}
    \end{frame}

    \begin{frame}{行財政改革}{}
        \begin{small}
            \begin{itemize}
                \setlength{\parsep}{.5mm}
                \setlength{\itemsep}{2mm}
                \item 特定の団体とのしがらみや古い規制に縛られず、維新の改革を推し進め、財政の健全化・兵庫の成長・住民サービスの拡充を実現します。\par
                またデジタル社会に対応した新しい行政の形を1日も早く実現し、利用者の多種多様な環境やニーズを踏まえたきめ細かいサービスを低コストで提供できる体制を構築します。
            \end{itemize}
        \end{small}
    \end{frame}

    \begin{frame}{行財政改革}{}
        \begin{small}
            \begin{itemize}
                \setlength{\parsep}{.5mm}
                \setlength{\itemsep}{2mm}
                \item 政治・行政・既得権の馴れ合い・ぬるま湯・もたれ合いで続いてきた古い政治を壊し、本気の改革を断行していくために、自ら議員報酬の一部を寄付するなど身を切る改革を徹底\par
                また議員定数削減、政務活動費の透明化や費用弁償、日当の見直し、知事・市長・町長の退職金の廃止・削減を求め、企業団体献金の受け取りも禁止
                \item 超少子高齢化・人口減少時代、急速に変化する国際情勢において経済構造の変容、人口減少・超高齢化、地域間格差拡大といった課題に対応し、住民生活を豊かにしていくための聖域なき行財政改革
                \item それぞれの地域が自らの努力と創意工夫によって財政的に自立することで、持続可能で自由度の高い予算編成ができる骨太な財政体質の実現
                \item 議会でのオンライン委員会の実施の推進
            \end{itemize}
        \end{small}
    \end{frame}

    \begin{frame}{行財政改革}{}
        \begin{small}
            \begin{itemize}
                \setlength{\parsep}{.5mm}
                \setlength{\itemsep}{2mm}
                \item より良い住民サービスを実現するため、意欲と能力のある職員を適材適所に抜擢する組織に変えていくべく抜本的な公務員改革の断行
                \item 退職した職員の再就職について、透明性・公平性をより一層高め、いわゆる「天下り」を根絶
                \item 産業振興や観光施策、公営住宅など未だに残る兵庫県と神戸市による非効率な二重行政を解消し、県市の連携強化・権限移譲を推進
                \item これまでの随意契約を見直し、公正な競争入札を実施
                \item 外郭団体は、民間で行い得る事業は民営化し、個別のミッションを明確にしたうえ、適切に業務を遂行できているかを評価・検証することにより、市民生活に不利益が生じると合理的に認められるもの以外は基本方針として全廃
            \end{itemize}
        \end{small}
    \end{frame}

    \begin{frame}{行財政改革}{}
        \begin{small}
            \begin{itemize}
                \setlength{\parsep}{.5mm}
                \setlength{\itemsep}{2mm}
                \item 教育委員会や外郭団体も含めた全庁において外部監査・内部監査・内部統制により指摘された事項は速やかに改善する体制を構築し、業務の適正かつ効果的な執行を確保
                \item 自治体行政におけるデジタルファーストの考え方の浸透\par
                また行政情報のビッグデータ化と、AI活用による分析、都市OSの整備
                \item ICT活用やDXの推進により、行政手続きのオンライン化、ペーパーレス化、リモート相談窓口など住民や事業者が区役所に来庁せずに済む環境作りの拡充\par
                住民目線に立った行政サービスの向上や業務効率化を推進
                \item インターネットの普及による高齢者や低所得世帯を中心としたデジタル・デバイド(情報格差)の解消に向け、共生型ネット社会を推進
            \end{itemize}
        \end{small}
    \end{frame}

    \begin{frame}{行財政改革}{}
        \begin{small}
            \begin{itemize}
                \setlength{\parsep}{.5mm}
                \setlength{\itemsep}{2mm}
                \item 働き方改革の一環として、RPAを積極的に導入し、業務自動化による生産性向上
                \item 住民サービスのさらなる向上を図るため、マイナンバーカードの普及を促進するとともに、個人情報の漏えいや、大阪の基幹病院で起こったようなサイバー攻撃などへの備えとして、国、県、民間事業者との連携強化し、情報セキュリティ対策を徹底
                \item 自治体システムにおける特定企業への随意契約を見直し、オープン化
                \item 女性の活躍と元気な高齢者の活用を促す「就労支援プログラム」の開発
                \item 役所の広報について、これまでの広報媒体の発信力強化とともに、専門家の知見も得ながらデジタル広報を活用したより効果的な発信
            \end{itemize}
        \end{small}
    \end{frame}

    \begin{frame}{行財政改革}{}
        \begin{small}
            \begin{itemize}
                \setlength{\parsep}{.5mm}
                \setlength{\itemsep}{2mm}
                \item ホームページは、アクセス解析を踏まえ、県民の視点に立ったコンテンツの充実、見やすく分かりやすい改善を適時行い、SNS、アプリ等を通じた双方向型住民コミュニケーションを促進
                \item 広聴において直接対話や、紙媒体、インターネット等あらゆる手法を用い、寄せられた意見や提言、県民ニーズを的確に把握し、各施策や業務改善に反映させることができる体制を構築
                \item 公共施設を適切かつ効率的に維持管理・更新するため、規模の最適化に向けて情報の一元化と将来計画の見直し。また施設の機能と必要性を十分に分析し、安全性と利便性の調和を考慮した複合化・多機能化
                \item 公設公営の施設管理や大型再整備については、政策的・投資的事業効果を適切に見極め、民間資金・活力を最大限活用しながら、最小の経費で最大の効果を得られるような取り組みを推進
            \end{itemize}
        \end{small}
    \end{frame}

\section{経済成長戦略と文化・スポーツ振興}
    \begin{frame}[plain, noframenumbering]{}{}
        \sectionpage
    \end{frame}
    
    \begin{frame}{経済成長戦略と文化・スポーツ振興}{}
        \begin{small}
            \begin{itemize}
                \setlength{\parsep}{.5mm}
                \setlength{\itemsep}{2mm}
                \item 異なる歴史文化や気候風土を有する旧五国からなる兵庫の多様性を活かした観光政策や高度なものづくり産業、地場産業、農林水産業の育成を強化し、地域経済の活性化と雇用を促進します。\par
                またコロナ禍において打撃を受けた兵庫経済の回復の為、県内中小企業者に対して、県市協調のうえ、雇用の維持と事業継続を下支えするきめ細やかな支援策を実施します。
            \end{itemize}
        \end{small}
    \end{frame}
    
    \begin{frame}{経済成長戦略と文化・スポーツ振興}{}
        \begin{small}
            \begin{itemize}
                \setlength{\parsep}{.5mm}
                \setlength{\itemsep}{2mm}
                \item これまでの企業を延命させる補助金行政による経済対策を見直し、企業の健全な切磋琢磨を通じて地域経済の活性化を実現
                \item コロナ禍において、業種・業態や事業規模に応じた支援策が必要であり、事業者に対して迅速に支援が行き届く必要があるため、新たな支援策の検討・既存支援策の期間延長や要件緩和、申請・事務手続きの簡素化等を国に要望
                \item ひょうご神戸スタートアップ・エコシステムコンソーシアムにおいては、京阪神の自治体、経済団体、大学、民間組織などの連携を強化しながら、グローバル拠点都市として、より多くの民間投資を喚起することができるスタートアップ支援\par
                また世界に羽ばたく起業家を育成するため、知的交流拠点施設「ANCHOR KOBE」との連携により、業界の垣根を超えたイノベーションを創出
            \end{itemize}
        \end{small}
    \end{frame}
    
    \begin{frame}{経済成長戦略と文化・スポーツ振興}{}
        \begin{small}
            \begin{itemize}
                \setlength{\parsep}{.5mm}
                \setlength{\itemsep}{2mm}
                \item 保証協会の体制強化、保証料の軽減など、企業が民間融資を受けやすい環境の整備
                \item 多様性のある兵庫の価値や地域の魅力を活かすため、発信する情報や媒体、ターゲットを明確にした発信を強化し、各自治体の都市ブランド力を向上
                \item 自然豊かな兵庫の強みを生かした観光や宿泊、ワーケーションなどの賑わい創出を進め、民間活力を活用した地域の活性化
                \item 県内の農業の担い手の確保、遊休農地の発生防止と解消、また農地の効率的な活用や、6次産業化、ロボット技術やAI・IoTなど先端技術の活用によるスマート農業など次世代型農林水産業を促進
            \end{itemize}
        \end{small}
    \end{frame}
    
    \begin{frame}{経済成長戦略と文化・スポーツ振興}{}
        \begin{small}
            \begin{itemize}
                \setlength{\parsep}{.5mm}
                \setlength{\itemsep}{2mm}
                \item 過疎地においても高速インターネットを不自由なく使える環境を整備し、企業の社員にワーケー ションを行い、農林水産業を副業で営める仕組みをつくることで、交流人口の拡大や定住を促進
                \item 「農家と消費者のための農業」への大転換、大改革\par
                生産性や品質を高める農業者を支援し、県内で生産される農林水産物のブランド戦略を強化\par
                「稼げる農業」を推進するとともに、中山間地や家族経営の農業の持続的な営みを可能とする農政を展開
                \item 農地法を改正し、株式会社をはじめとしたあらゆる主体による土地所有や新規参入を促進することによる農業の活性化\par
                高齢化・担い手不足対策として、若者が独立就農して稼げる農業経営ができるよう、新規就農を促進
                \item 森林の適正な保全による国産材の需要を拡大し、県産木材の積極的な活用を支援
            \end{itemize}
        \end{small}
    \end{frame}
    
    \begin{frame}{経済成長戦略と文化・スポーツ振興}{}
        \begin{small}
            \begin{itemize}
                \setlength{\parsep}{.5mm}
                \setlength{\itemsep}{2mm}
                \item 鳥獣害対策については、捕獲等への支援を行うことにより個体数を管理するとともに、ジビエとしての加工・流通・販売のための衛生管理の高度化\par
                あわせて適切な対策ができるきる専門人材の育成および集落ぐるみの取り組みを後押しし、スマートセンサーやドローン等の新技術導入を促進
                \item 農業生産条件不利地域の農林漁業者に対する最低所得補償制度(直接払い)の導入促進
                \item 兵庫県内の風土を活かした有機農業を支援するため、普及指導の体制整備や有機農業者の人材育成、有機 JAS 認証を取得しやすい環境づくり\par
                また、需要拡大に向けた販路の多様化や消費者への理解と関心の増進を図るなど有機農業の生産拡大を推進
            \end{itemize}
        \end{small}
    \end{frame}
    
    \begin{frame}{経済成長戦略と文化・スポーツ振興}{}
        \begin{small}
            \begin{itemize}
                \setlength{\parsep}{.5mm}
                \setlength{\itemsep}{2mm}
                \item 県内のJR赤字路線に関して、路線の維持・存続に向けて、周辺自治体とともに国への関与を強め、利用促進と利便性向上を図りながら沿線を活性化
                \item 大型放射光施設SPring-8をはじめとする播磨科学公園都市、世界最速のスーパーコンピューター「富岳」を擁する理化学研究所・神戸医療産業都市など、世界的な科学技術基盤の集積を推進\par
                国内外の高度人材・資本を呼び込み、産官学連携による基幹産業を強化
                \item 最先端の研究開発を支える国際水準の大学や研究機関等が多数立地する強みを活かし、今後、発展が予測される業界や成長分野を中心に戦略的な企業誘致と、新たな雇用・税収を創出
                \item 医療産業都市神戸の国際競争力をさらに高めるため、アフターコロナ時代を見据え、関連企業や海外主要クラスターとの交流・連携を促進
            \end{itemize}
        \end{small}
    \end{frame}
    
    \begin{frame}{経済成長戦略と文化・スポーツ振興}{}
        \begin{small}
            \begin{itemize}
                \setlength{\parsep}{.5mm}
                \setlength{\itemsep}{2mm}
                \item 神戸港の生産性向上のため、港湾におけるIT化、さらにはDXを積極的に活用し、官民連携による神戸港の国際競争力強化
                \item アフターコロナ時代を見据えたクルーズ船の国内外客の観光・滞在を促すためにも県内主要港における着地型観光を推進し、地域経済への利益創出・還元
                \item 国際機関や領事館、MICEに関しては、都市ブランド向上、経済活性化の観点から兵庫県、神戸市、経済団体、観光局(DMO)が一体となり、役割分担・機能強化の方向性を定める戦略的な誘致を推進
                \item 兵庫の伝統芸能や音楽など文化振興を図り、観光資源、経営資源として国内外に発信する取り組みを推進。県内の歴史的建造物を活用した文化活動を支援
            \end{itemize}
        \end{small}
    \end{frame}
    
    \begin{frame}{経済成長戦略と文化・スポーツ振興}{}
        \begin{small}
            \begin{itemize}
                \setlength{\parsep}{.5mm}
                \setlength{\itemsep}{2mm}
                \item KOBE2024世界パラ陸上競技選手権大会、2027年ワールドマスターズゲームズ関西の機運醸成に努めるとともに、大会成功に向けたボランティア運営業務や宿泊の受け入れ体制の充実\par
                世界各国から多数の競技者や関係者が来県されることから、県内の周遊促進施策に取り組むとともに、子供たちや地域住民との交流機会を設けたスポーツ振興施策推進と地域経済の活性化
                \item 社会人が職業上の新たな知識・技術を習得し、日常生活における教養を身につけるなど、何度でも学び続けることができるリカレント教育の充実
            \end{itemize}
        \end{small}
    \end{frame}

\section{教育・子育て施策のさらなる充実}
    \begin{frame}[plain, noframenumbering]{}{}
        \sectionpage
    \end{frame}

    \begin{frame}{教育・子育て施策のさらなる充実}{}
        \begin{small}
            \begin{itemize}
                \setlength{\parsep}{.5mm}
                \setlength{\itemsep}{2mm}
                \item 超少子高齢化・人口減少時代にあって、子育てをする現役世代に徹底した投資を行い、子どもを産み育てやすい社会を作っていくことは、今や国家の最重要課題であります。\par
                子どもたちが国や文化の違いを乗り越えて力強く未来を生き抜く力を備えるための教育改革に全力で取り組み、「日本一子どもを産み育てやすい兵庫」を目指します。
            \end{itemize}
        \end{small}
    \end{frame}

    \begin{frame}{教育・子育て施策のさらなる充実}{}
        \begin{small}
            \begin{itemize}
                \setlength{\parsep}{.5mm}
                \setlength{\itemsep}{2mm}
                \item 私学を含めた県内全域での教育無償化の推進。地域間格差の解消
                \item 県内全域での温かくて美味しい全員喫食を実現と、小中学校給食の無償化、食育の推進
                \item 教員の業務負担の軽減や保護者の利便性向上、徴収・管理業務の効率化、透明性の向上、不正防止等を図るため、無償化が実現するまでの間、学校給食費の公会計化を促進
                \item 家庭の経済状況による教育格差の是正と、子育て世帯の負担軽減を図る教育バウチャー(塾代助成事業)を実現するための制度設計や環境整備の促進\par
                学校外教育への支援を拡充
            \end{itemize}
        \end{small}
    \end{frame}

    \begin{frame}{教育・子育て施策のさらなる充実}{}
        \begin{small}
            \begin{itemize}
                \setlength{\parsep}{.5mm}
                \setlength{\itemsep}{2mm}
                \item 不登校児童生徒の対策強化に関わる担い手を増やし、当該児童生徒が行う多様な学習活動の実情を踏まえ、フリースクールの運営者や保護者との意見交換の拡充\par
                またオンラインを活用した授業の配信など個々の状況に応じた対策強化や、不登校特例校の設置
                \item 学校教育へのICT機器の活用による個別最適化された学習支援、またインターネット等を利用した情報活用教育など現代社会に対応した教育内容のアップデートを含むGIGAスクール構想の推進
                \item 子どもたち一人一人の個性を伸ばす多様な学びや、主体的、共創的な学びの実効性を高めるため、OECD諸国の中でも突出して多い1クラスあたりの生徒数について見直しを行い、少人数制学級を実現
                \item 児童生徒や教員へのSNSを含めた情報モラル教育の徹底。ネットいじめや有害サイトへのアクセスなど、実際にトラブルになる可能性の高い内容を精査し、教育委員会が基本方針を示したうえで全校での予防策を強化
                \item 教員の負担軽減を図るため、校務分掌や部活動の見直し。ICT活用による校務の効率化など、他防火対策・働き方改革を促進。指導力を強化し、教育に専念できる体制を構築
            \end{itemize}
        \end{small}
    \end{frame}

    \begin{frame}{教育・子育て施策のさらなる充実}{}
        \begin{small}
            \begin{itemize}
                \setlength{\parsep}{.5mm}
                \setlength{\itemsep}{2mm}
                \item 不登校児童生徒の対策強化に関わる担い手を増やし、当該児童生徒が行う多様な学習活動の実情を踏まえ、フリースクールの運営者や保護者との意見交換の拡充\par
                またオンラインを活用した授業の配信など個々の状況に応じた対策強化や、不登校特例校の設置
                \item 学校教育へのICT機器の活用による個別最適化された学習支援、またインターネット等を利用した情報活用教育など現代社会に対応した教育内容のアップデートを含むGIGAスクール構想の推進
                \item 子どもたち一人一人の個性を伸ばす多様な学びや、主体的、共創的な学びの実効性を高めるため、OECD諸国の中でも突出して多い1クラスあたりの生徒数について見直しを行い、少人数制学級を実現
            \end{itemize}
        \end{small}
    \end{frame}

    \begin{frame}{教育・子育て施策のさらなる充実}{}
        \begin{small}
            \begin{itemize}
                \setlength{\parsep}{.5mm}
                \setlength{\itemsep}{2mm}
                \item 児童生徒や教員へのSNSを含めた情報モラル教育の徹底\par
                ネットいじめや有害サイトへのアクセスなど、実際にトラブルになる可能性の高い内容を精査し、教育委員会が基本方針を示したうえで全校での予防策を強化
                \item 教員の負担軽減を図るため、校務分掌や部活動の見直し\par
                ICT活用による校務の効率化など、他防火対策・働き方改革を促進\par
                指導力を強化し、教育に専念できる体制を構築
                \item 子育て関係の申請手続きは、早急にオンライン化を進め、スマホで完結できる電子申請システムを構築
            \end{itemize}
        \end{small}
    \end{frame}

    \begin{frame}{教育・子育て施策のさらなる充実}{}
        \begin{small}
            \begin{itemize}
                \setlength{\parsep}{.5mm}
                \setlength{\itemsep}{2mm}
                \item 学校内で生じる問題の解決について、児童生徒本人への聞き取りを強化するとともに、臨床心理士・公認心理師を始めとする常勤スクールカウンセラーやオンラインカウンセラーの配置を県内全域で促進\par
                子どもの視点と専門的知見の双方から、いじめ・ヤングケアラー・不登校など多様化する子どもの悩みに対応できる体制の整備・強化
                \item グローバル人材の育成のため、ALT(外国語指導助手)の授業をオンライン化するなど実践的な英語の語学力育成や国際性を育む教育を推進
                \item 児童虐待や経済的環境等、様々な理由で社会的養護を必要とする子どものため、弁護士等の専門家を常駐させるなど児童相談所の機能強化とニーズに応じた機能分担の推進\par
                また特別養子縁組の促進や里親委託率の向上のため、自治体や民間支援団体との連携強化
            \end{itemize}
        \end{small}
    \end{frame}

    \begin{frame}{教育・子育て施策のさらなる充実}{}
        \begin{small}
            \begin{itemize}
                \setlength{\parsep}{.5mm}
                \setlength{\itemsep}{2mm}
                \item 学校施設の老朽化等における事故防止を図るため、必要な安全点検を強化するとともに、教育委員会、建築関連局、民間の専門家等の役割分担を明確にしたうえ、児童生徒が安心・安全に学校生活を送れるような環境整備
                \item 隠れ待機児童も含めた待機児童解消のため、保護者のニーズに合った保育所の整備だけでなく、保育士不足解消と質の低下を打開するための処遇改善を行い、各地域に見合った制度設計を構築
                \item 仕事と子育ての両立を支援するため、認可保育所を中心に多様な保育サービスの拡充とともに、保育の質を向上
                \item 長時間労働、サービス残業、持ち帰り残業を撤廃するなど保育士の働き方改革を推進による保育士不足を解消
            \end{itemize}
        \end{small}
    \end{frame}

    \begin{frame}{教育・子育て施策のさらなる充実}{}
        \begin{small}
            \begin{itemize}
                \setlength{\parsep}{.5mm}
                \setlength{\itemsep}{2mm}
                \item 一時保護や入所措置等に係るこどもの意見表明権を保障するため、弁護士を意見表明支援員として、子どもの意見を代弁する制度(アドボカシー)を構築
                \item 深刻化している児童虐待については、虐待通告への初期対応や児童の安全確保体制の強化および児童・保護者へのケア・指導体制の強化\par
                児童相談職員の相談窓口の人員増員、カウンセラーの充実とともに福祉、保険、医療、教育の各機関と連携によるきめ細やかな支援を実施
                \item 妊婦健康診査費用について、全ての妊婦が安心・安全な出産ができるよう無料化をはじめとした出産費用の軽減\par
                また、国における出産の健康保険適用を前提とした出産費用の完全無償化
                \item 子ども医療費の18歳までの無償化
            \end{itemize}
        \end{small}
    \end{frame}

\section{医療・福祉体制の充実}
    \begin{frame}[plain, noframenumbering]{}{}
        \sectionpage
    \end{frame}

    \begin{frame}{医療・福祉体制の充実}{}
        \begin{small}
            \begin{itemize}
                \setlength{\parsep}{.5mm}
                \setlength{\itemsep}{2mm}
                \item 超高齢化社会の進展に伴い、自治体における扶助費負担の見通しは今後も右肩上がりで推移していくことが予測されている時代にあって、全ての人に手厚い福祉を届けようとすることは制度そのものの崩壊に繋がりかねません。\par
                限りある財源の中で、真に必要な人に確実に福祉の手が行き届くための選択と集中によって、力強い福祉政策の実現を目指します。
            \end{itemize}
        \end{small}
    \end{frame}

    \begin{frame}{医療・福祉体制の充実}{}
        \begin{small}
            \begin{itemize}
                \setlength{\parsep}{.5mm}
                \setlength{\itemsep}{2mm}
                \item 新型コロナウイルスの感染拡大と想定外のリスクを踏まえ、各自治体、医師会、初期・二次・三次医療機関、宿泊療養施設等々の機能分担・連携を進め、医療ひっ迫時における新たな医療提供体制を確立
                \item コロナ感染拡大にともなう保健所の負担軽減を図るため、保健師の増員や、AIやIoTなどデジタル化の推進\par
                入院調整や健康観察、搬送業務を円滑化
                \item 公立病院において現場の業務効率化、遠隔診療(オンライン診療)の実用化、医療ネットワークの構築、データのクラウド化など医療DXを推進
            \end{itemize}
        \end{small}
    \end{frame}

    \begin{frame}{医療・福祉体制の充実}{}
        \begin{small}
            \begin{itemize}
                \setlength{\parsep}{.5mm}
                \setlength{\itemsep}{2mm}
                \item 地域包括ケアシステムの構築、特別養護老人ホームの整備など医療・介護・予防・住まい、生活支援の一体的な提供を行う地域包括ケアシステムの構築。地域の特性・実情に応じたきめ細やかなサービス提供体制を整備
                \item 社会保険への過度な税投入を見直し、高齢者の健康寿命の延伸による医療費縮減を推進\par
                また地域包括支援センターや自治会、老人会などの連携を密にし、介護予防・フレイル予防の一体的な実施
                \item 認知症予防対策のため、MCI(軽度認知障害)予防施策の拡充と、早期発見・早期支援
            \end{itemize}
        \end{small}
    \end{frame}

    \begin{frame}{医療・福祉体制の充実}{}
        \begin{small}
            \begin{itemize}
                \setlength{\parsep}{.5mm}
                \setlength{\itemsep}{2mm}
                \item シニア世代の一人ひとりの意思や能力に応じた多様な就労・ボランティア・生涯学習を選択できる環境整備と、年齢に関係なく活躍できる社会を推進
                \item 生活保護の不正受給を排除するとともに、稼働年齢層の就職支援を強化し、扶養義務者がいる場合はその義務を果たさせる等による適正な運用
                \item 生活保護費の約半分を占める医療扶助の適正化も喫緊の課題であり、特に電子レセプトデータの徹底した点検を行うとともに、頻回受診や重複受診者に対する適正受診指導などの取り組みを強化し、さらなる医療扶助の適正化
                \item 子どもの貧困対策、ヤングケアラーへの支援のための実態調査を強化
            \end{itemize}
        \end{small}
    \end{frame}

    \begin{frame}{医療・福祉体制の充実}{}
        \begin{small}
            \begin{itemize}
                \setlength{\parsep}{.5mm}
                \setlength{\itemsep}{2mm}
                \item 公共施設・公共交通及び道路のバリアフリー化を促進すると共に、電動車いすの貸し出しなど、障がい者、高齢者などが利用しやすいサービスの充実と、障がい者でも健常者と同じく平等に生活できるノーマライゼーション社会の実現
                \item 発達障害者支援センターによる関係機関との連携強化や、早期支援・早期療育体制の構築、特別支援教育の充実、就労支援の充実など、ライフステージに応じた乳幼児期から成人期までの一貫した支援強化
                \item 人と動物が共存できる生活環境作りのため、飼い主・販売主・市民の意識の向上、引き取り数の削減\par
                返還と適正譲渡の推進に関する具体的な目標を立て、実行管理をしたうえ、犬・猫の理由なき殺処分ゼロを目標
            \end{itemize}
        \end{small}
    \end{frame}

\section{環境保全とゴミ問題、エネルギー政策}
    \begin{frame}[plain, noframenumbering]{}{}
        \sectionpage
    \end{frame}

    \begin{frame}{環境保全とゴミ問題、エネルギー政策}{}
        \begin{small}
            \begin{itemize}
                \setlength{\parsep}{.5mm}
                \setlength{\itemsep}{2mm}
                \item 自然豊かな兵庫の森林、里山、川、海を再生・保全する取り組みを強化し、人と自然が共生する地域づくりを進めるとともに、\par
                次世代に向けたエネルギー政策を推進し、2050年カーボンニュートラル(排出量から吸収量と除去量を差し引いた合計をゼロにする)に向けた取り組みを強化します。
            \end{itemize}
        \end{small}
    \end{frame}

    \begin{frame}{環境保全とゴミ問題、エネルギー政策}{}
        \begin{small}
            \begin{itemize}
                \setlength{\parsep}{.5mm}
                \setlength{\itemsep}{2mm}
                \item 自然環境と生物多様性の保全・再生に向け、持続可能な活動が展開できるよう、多様な主体の参加と地域住民の協働を促進
                \item 食品ロス削減のため、市民、食品製造業・加工業、卸売業、小売業、飲食業の皆さんができることを改めて精査し、各事業者と連携した啓発キャンペーンやフードドライブの実施を推進
                \item ごみ収集業務については、長期的には事業全体の民間委託を踏まえ、まずは事業所ごとに段階的に進め、人件費の抑制と収集体制を効率化
            \end{itemize}
        \end{small}
    \end{frame}

    \begin{frame}{環境保全とゴミ問題、エネルギー政策}{}
        \begin{small}
            \begin{itemize}
                \setlength{\parsep}{.5mm}
                \setlength{\itemsep}{2mm}
                \item 収集業務の担い手となっている許可業者数の拡大や入札参加資格要件の緩和など、より競争性を高める手法を取り入れ、ごみ処理にかかる経費削減とサービスの向上
                \item 不法投棄を削減するため、公営の監視カメラの適切な運用。また日常の監視パトロール活動、県警、地元自治会等との連携を強化し、撲滅に向けた実行性のある取り組みを推進
                \item 地域に暮らす外国人住民の中には、生活習慣や言語・文化の違いから様々な不安を抱えて生活しており、ゴミの分別方法や大型ゴミの出し方など、多様な手法を用いたルールやマナーの啓発活動を促進
            \end{itemize}
        \end{small}
    \end{frame}

    \begin{frame}{環境保全とゴミ問題、エネルギー政策}{}
        \begin{small}
            \begin{itemize}
                \setlength{\parsep}{.5mm}
                \setlength{\itemsep}{2mm}
                \item 低炭素社会を先導する都市づくりとして、電気自動車(EV)、燃料電池自動車(FCV)、プラグインハイブリッド車(PHV)等などの普及促進\par
                また快適で環境にやさしい建築物の誘導を行う制度と、地球にやさしい都市づくり
                \item 国における2050年カーボンニュートラル宣言の実現に向け、県内市町で実行性のある施策の具現化を推進
                \item 太陽光、風力、地熱、バイオマス等の再生可能エネルギーの導入については、障害となる規制を国で見直した上、地域社会がうるおう仕組みづくりの構築
            \end{itemize}
        \end{small}
    \end{frame}

\section{道路・交通・公園等のインフラ整備の充実}
    \begin{frame}[plain, noframenumbering]{}{}
        \sectionpage
    \end{frame}

    \begin{frame}{道路・交通・公園等のインフラ整備の充実}{}
        \begin{small}
            \begin{itemize}
                \setlength{\parsep}{.5mm}
                \setlength{\itemsep}{2mm}
                \item ミッシングリンクの解消を含めた道路ネットワークの整備を強力に進め、主要都市間や県内各地の渋滞箇所の解消を図り、人とモノの交流を促進します。\par
                水道事業は、節水型社会や人口減少社会の進展による中長期的な水需要の減少を踏まえ、経営の効率化による経営の健全化を推進し、安全・良質な水を安定的に公正な料金で継続的に提供します。
            \end{itemize}
        \end{small}
    \end{frame}

    \begin{frame}{道路・交通・公園等のインフラ整備の充実}{}
        \begin{small}
            \begin{itemize}
                \setlength{\parsep}{.5mm}
                \setlength{\itemsep}{2mm}
                \item 県民の日常生活や地域課題に密接に関わる幹線道路の渋滞解消・緩和を効率的かつ効果的に進めるとともに、地域の防災性を高め、都市機能の向上
                \item 関西3空港懇談会における神戸空港、関西国際空港、大阪国際(伊丹)空港の一体的な航空戦略のもと、役割の最適化\par
                また神戸空港の国際化に向け、四国、中国方面を含む神戸市以西の新たな空港利用者の開拓に努め、関西全体の需要拡大への貢献
                \item 地域の実情に応じた路線バス、オンデマンドバス、コミュニティバス、乗り合いタクシー等を適切に組み合わせ、高齢者をはじめとする交通弱者の「生活の足」の維持確保\par
                また持続可能な移動手段を維持するため、運行事業への参画や啓発など地域住民の利用を促進
            \end{itemize}
        \end{small}
    \end{frame}

    \begin{frame}{道路・交通・公園等のインフラ整備の充実}{}
        \begin{small}
            \begin{itemize}
                \setlength{\parsep}{.5mm}
                \setlength{\itemsep}{2mm}
                \item 市民に身近な公園については、コロナ禍や地域の実情に応じた遊具や健康器具等の整備の充実\par
                施設に不具合や危険な状況を発見した場合は、迅速・適切に処置を実施し、公園利用者の安全確保
                \item 都市公園については、アーバンスポーツ施設の環境整備を検討し、官民連携のうえ、民間のノウハウを積極的に取り入れた取り組みを推進
                \item 公園トイレ改築事業(トイレチェンジアクション)は、住民の利用ニーズを考慮に入れ、民間事業者の技術やノウハウを活かした改善を推進
            \end{itemize}
        \end{small}
    \end{frame}

    \begin{frame}{道路・交通・公園等のインフラ整備の充実}{}
        \begin{small}
            \begin{itemize}
                \setlength{\parsep}{.5mm}
                \setlength{\itemsep}{2mm}
                \item 災害時の被害を最小限に抑えるための無電柱化・共同溝建設及び下水管の耐震化を促進
                \item 橋梁や堤防、道路、トンネルなどインフラの老朽化対策について、ドローンやAI、IoTなど最新技術の活用によるメンテナンスの高度化・効率化など継続的な推進。インフラの集約や不要な施設を撤廃
                \item 水道事業はICT利活用による業務の効率化や、AI技術を利用した施設維持管理など最先端の新技術をより積極的に活用し、人件費削減による組織のスリム化\par
                スマートメーターの全戸導入に向け、価格の低減を促進するとともに、業務プロセスのDX化を推進
            \end{itemize}
        \end{small}
    \end{frame}

\section{防災、危機管理体制の強化と、治安の向上}
    \begin{frame}[plain, noframenumbering]{}{}
        \sectionpage
    \end{frame}

    \begin{frame}{防災、危機管理体制の強化と、治安の向上}{}
        \begin{small}
            \begin{itemize}
                \setlength{\parsep}{.5mm}
                \setlength{\itemsep}{2mm}
                \item 住民生活の安心安全を担保するため、阪神・淡路大震災の教訓やノウハウを活かし、地域防災力を高める取り組みを強化していきます。\par
                また住民の安全で安心な日常生活を守るための治安のさらなる向上も重要な課題であり、防犯カメラの増設や、特殊詐欺をはじめとした新たな犯罪から身を守るための政策を推進します。
            \end{itemize}
        \end{small}
    \end{frame}

    \begin{frame}{防災、危機管理体制の強化と、治安の向上}{}
        \begin{small}
            \begin{itemize}
                \setlength{\parsep}{.5mm}
                \setlength{\itemsep}{2mm}
                \item 未曾有の災害に備え、防災マニュアルや兵庫県地域防災計画、南海トラフ地震防災対策推進計画の策定を適時、見直し。\par
                また災害発生時に迅速かつ安全な避難や災害弱者へきめ細かい対応ができるよう、阪神・淡路大震災の教訓を活かした地域の防災組織の機能強化
                \item 避難先は体育館のみならず、公営住宅の空き家活用、民間賃貸住宅の積極的活用といった要配慮者に対しての備えや、仮設住宅の設置などに関しては広域的な視点から県市の連携を進めた防災機能を強化
                \item 避難所での授乳スペースや更衣室の確保、また女性用品の適切な配布など、女性に配慮した避難所のあり方の見直し。\par
                また、災害時には女性や子供に対する犯罪が起こるリスクが高まることから、防犯意識を高める取り組みを推進
            \end{itemize}
        \end{small}
    \end{frame}

    \begin{frame}{防災、危機管理体制の強化と、治安の向上}{}
        \begin{small}
            \begin{itemize}
                \setlength{\parsep}{.5mm}
                \setlength{\itemsep}{2mm}
                \item 学校やコミュニティーセンター、公園など公的施設の避難所について、災害発生時における運営を円滑に進めるため、備品・備蓄の充実や、通信手段に必要となる非常用電源の確保等インフラ面や耐震対策を強化
                \item 災害発生時には避難所情報はもとより、水道・電気などのライフラインに関する情報や、学校園をはじめとした行政施設の運営状況など市民に必要となる多種多様な情報を正確かつ適切なタイミングで発信。\par
                SNSを活用したプッシュ型の発信や、HPの表示順序を変えるなどのプル型情報への適切な誘導など、ICTを効果的に活用した情報発信の仕組みを充実
                \item 防災コミュニティの高齢化に伴い、実態の把握や若い世代の啓発と後継者育成
            \end{itemize}
        \end{small}
    \end{frame}

    \begin{frame}{防災、危機管理体制の強化と、治安の向上}{}
        \begin{small}
            \begin{itemize}
                \setlength{\parsep}{.5mm}
                \setlength{\itemsep}{2mm}
                \item 消防局及び消防団のIT化・DX化について、マイナンバーカードを活用した救急業務の迅速化、救急現場からの情報の取得・デジタル映像の保存と検証の実施、事務処理の作業効率化、デジタル人材の育成などを推進
                \item ドローンを活用した災害監視体制の強化と、県民への迅速かつ適切な情報発信
                \item 橋梁や堤防などのインフラ設備の老朽化対策や、豪雨災害を想定した排水設備等を整備
                \item 弾道ミサイルの爆風などから直接の被害を軽減する対処として、避難経路や緊急一時避難施設の確保・充実\par
                Jアラート(全国瞬時警報システム)が発動した際の取るべき行動について、国とも連携した県民への周知徹底
            \end{itemize}
        \end{small}
    \end{frame}

    \begin{frame}{防災、危機管理体制の強化と、治安の向上}{}
        \begin{small}
            \begin{itemize}
                \setlength{\parsep}{.5mm}
                \setlength{\itemsep}{2mm}
                \item 災害時の被害を最小限に抑えるための無電柱化・共同溝建設及び下水管の耐震化
                \item 防犯カメラの増設や活用するシステム整備、また見守り活動やスマホ等を利用した啓発活動の充実\par
                また県警や地域住民との連携を着実に進め、犯罪被害の防止や発生時の迅速な対応強化
                \item 超高齢化が進む中、消費者が悪質商法のターゲットにされないよう訪問販売や電話勧誘販売\par
                またSNS等のインターネットを通じた通信販売の勧誘等に関する被害に対処するための実効性のある対策、啓発を促進
                \item 新設する消防署に関しては、交流・学び・憩い・賑わいの観点から、住民が気軽に利用でき地域に愛される開かれた空間の有効活用と複合施設としての整備
            \end{itemize}
        \end{small}
    \end{frame}

\section{地方分権による地方の自立と、\\広域連携の強化による兵庫・関西の発展}
    \begin{frame}[plain, noframenumbering]{}{}
        \sectionpage
    \end{frame}

    \begin{frame}{地方分権による地方の自立と、\\広域連携の強化による兵庫・関西の発展}{}
        \begin{small}
            \begin{itemize}
                \setlength{\parsep}{.5mm}
                \setlength{\itemsep}{2mm}
                \item 国が成熟するとともに、中央官庁による強力な岩盤規制が地方の自由な裁量による行政運営を阻害し、地方の自立を妨げています。\par
                東京一極集中の加速と国力の減退を招いている現状に鑑み、地方が自由な裁量において行政運営を行う地方分権型国家への国家構造の転換を強力に推進します。\par
                また関西広域連合との連携により、防災・観光・医療・産業振興・農林水産振興・環境保全・文化・スポーツ振興等の深化を図ります。
            \end{itemize}
        \end{small}
    \end{frame}

    \begin{frame}{地方分権による地方の自立と、\\広域連携の強化による兵庫・関西の発展}{}
        \begin{small}
            \begin{itemize}
                \setlength{\parsep}{.5mm}
                \setlength{\itemsep}{2mm}
                \item 国、都道府県、市町の役割分担の明確化\par
                都道府県と政令市における二重行政、二元行政の解消\par
                将来的な道州制を見据えた広域機能の集権化
                \item 市町村間の行政連携による効率化、都市内分権の推進による住民自治を拡充
                \item 医療や農林水産業、観光、都市再生、また税制・金融支援など国家戦略特区を活用した規制緩和\par
                県内のあらゆる産業の国際競争力の強化や地域創生
            \end{itemize}
        \end{small}
    \end{frame}

    \begin{frame}{地方分権による地方の自立と、\\広域連携の強化による兵庫・関西の発展}{}
        \begin{small}
            \begin{itemize}
                \setlength{\parsep}{.5mm}
                \setlength{\itemsep}{2mm}
                \item 新型コロナウイルス感染症対策も含めた広域緊急医療体制を強化
                \item 関西の歴史、文化、自然や食など多様な強みを活かした広域観光周遊ルートの創出を促進
                \item 2025年に開催される大阪・関西万博の兵庫への経済波及効果を高めるため、万博の企画・運営に関する県内事業者への受注機会の確保\par
                また世界から注目を集めるイベントの開催や、空飛ぶクルマの離発着場の設置、夢洲会場と神戸港や淡路島を結ぶ海上アクセス整備事業を推進し、播磨灘・大阪湾ベイエリアを活性化
            \end{itemize}
        \end{small}
    \end{frame}
\end{document}