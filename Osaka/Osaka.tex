\documentclass[dvipdfmx]{beamer}
%テーマ
\usetheme[numbering=counter]{corporate}
%フォント
\renewcommand{\kanjifamilydefault}{\gtdefault}
%パッケージ
\usepackage{pxjahyper}
\usepackage{tikz}
\usepackage{xcolor}
%色設定
\definecolor{IshinDark}{RGB}{0,85,46}
\definecolor{IshinMain}{RGB}{183,204,71}
\definecolor{IshinBack}{RGB}{38,191,0}
\setbeamercolor*{palette tertiary}{fg=IshinDark, bg=IshinDark}
\setbeamercolor*{palette primary}{fg=IshinBack,bg=IshinBack}
\setbeamercolor*{palette secondary}{fg=IshinMain,bg=IshinMain}

%タイトル設定
\title{2023 大阪府市政策集}
\subtitle{大阪を前へ、維新は挑戦をやめない}
\author{大阪維新の会}
\date[2023]{2023}

\begin{document}

\maketitle

\begin{frame}{目次}
    \tableofcontents
\end{frame}

\begin{frame}{はじめに}{}
    \begin{footnotesize}
        \begin{itemize}
            \setlength{\itemsep}{2mm}
            \item 大阪府市においては、大阪維新の会の知事・市長による府市連携による取り組みによって、様々な改革・施策を実現してきた結果、大きな税収の伸びを見せる等、各種の指標において、その成長が実感できる状況となっています。
            \item 大阪府市は、こうした成長戦略の実行もさることながら、大阪府の減債基金の復元や大阪市の市債残高の削減など、行財政改革の徹底による財政健全化も進めており、少子高齢化社会を見通し、負担を未来に先送りしない、希望の持てる都市を目指しています。大阪の改革と成長を実現することにより生まれた財源で、今まで無かった様々な住民サービスの拡充を続け、誰もが住みたいと思える大阪へ向けて着実に邁進しております。
            \item 次の挑戦として、子育て・教育にお金がかからない大阪を作るために、維新版教育無償化モデルの実現をかかげます。子どもには平等に教育・支援がいき渡ることを目指します。同時に子育て層を大阪に呼び込み、活気と財源を作ることにより、健康福祉を含めた大阪の住民サービスを大幅に底上げしていきます。 
        \end{itemize}
    \end{footnotesize}
\end{frame}

\begin{frame}{はじめに}{}
    \begin{small}
        \begin{itemize}
            \setlength{\itemsep}{2mm}
            \item また府内・府外の党勢拡大により府内及び近隣都道府県も含めた行政の最適化に着手できる段階にきており、新たな改革ステージにきたと考えています。大阪の府市一体条例により、大きな方向性を大阪府・大阪市の首長が共有しながら、行政運営を行い、大阪を成長させるとともに、その成長の果実を、住民に身近な特別区における住民サービスの拡充につなげていきたいと考えます。 大阪を元気にし、皆様のくらしを守り、充実させていくために、以下の3つの重点項目を立てたマニフェストを示し、来るべき大阪府知事・市長選挙に挑みます。
            \begin{enumerate}
                \item 日本一の子育て・教育サービスの実現
                \item 府市一体の成長戦略
                \item 大阪・関西万博の成功へ
            \end{enumerate}            
        \end{itemize}
    \end{small}
\end{frame}

\section{重点政策}
    \begin{frame}[plain,noframenumbering]
        \sectionpage
    \end{frame}

    \begin{frame}{重点政策}{}
        \begin{small}
            \begin{itemize}
                \setlength{\itemsep}{2mm}
                \item 大阪府市のみならず、府内及び近隣都道府県との連携も視野に入れながら、首都圏に対抗できる副首都圏を確立していきます。この間の改革・成長戦略により、特に大阪市内においてさらなる住民サービスの拡充を可能とする財源の確保がみえてきています。
                \item 大阪を持続的に発展させ、住民サービスの維持・拡充を進めるためには、大阪の人口増加を加速化させる、制度設計・インフラ整備・住民サービスの拡充である必要があります。
                \item 日本から、世界から選んでもらえる都市設計を目指し、そのための重点施策であります。
            \end{itemize}
        \end{small}
    \end{frame}

    \begin{frame}{日本一の子育て・教育サービスの実現}{維新版教育無償化モデルの実現}
        \begin{small}
            \begin{itemize}
                \setlength{\itemsep}{2mm}
                \item 0歳~2歳の保育料無償に関する所得制限撤廃
                \hspace{.2mm}
                \begin{tikzpicture}
                    \node[circle,white,fill=IshinDark,text centered,inner sep=1pt]{\scriptsize 市};
                \end{tikzpicture}
                \item 塾代助成の所得制限の撤廃
                \hspace{.2mm}
                \begin{tikzpicture}
                    \node[circle,white,fill=IshinDark,text centered,inner sep=1pt]{\scriptsize 市};
                \end{tikzpicture}
                \item 私立高校授業料無償化の所得制限撤廃
                \hspace{.2mm}
                \begin{tikzpicture}
                    \node[circle,white,fill=IshinMain,text centered,inner sep=1pt]{\scriptsize 府};
                \end{tikzpicture}
                \item 公立高校授業料無償化の所得制限撤廃
                \hspace{.2mm}
                \begin{tikzpicture}
                    \node[circle,white,fill=IshinMain,text centered,inner sep=1pt]{\scriptsize 府};
                \end{tikzpicture}
                \item 大阪公立大学授業料の所得制限撤廃
                \hspace{.2mm}
                \begin{tikzpicture}
                    \node[circle,white,fill=IshinMain,text centered,inner sep=1pt]{\scriptsize 府};
                \end{tikzpicture}
                \item 大阪公立大学大学院無償化の所得制限撤廃
                \hspace{.2mm}
                \begin{tikzpicture}
                    \node[circle,white,fill=IshinMain,text centered,inner sep=1pt]{\scriptsize 府};
                \end{tikzpicture}
            \end{itemize}
        \end{small}
    \end{frame}


    \begin{frame}{日本一の子育て・教育サービスの実現}{日本一の充実した子育て教育サービス}
        \begin{small}
            \begin{itemize}
                \setlength{\itemsep}{2mm}
                \item 保険適用で賄えない不妊治療への助成
                \hspace{.2mm}
                \begin{tikzpicture}
                    \node[circle,white,fill=IshinDark,text centered,inner sep=1pt]{\scriptsize 市};
                \end{tikzpicture}
                \item インフルエンザワクチンの子育て世帯への助成
                \hspace{.2mm}
                \begin{tikzpicture}
                    \node[circle,white,fill=IshinDark,text centered,inner sep=1pt]{\scriptsize 市};
                \end{tikzpicture}
                \item 子育てクーポンの実現
                \hspace{.2mm}
                \begin{tikzpicture}
                    \node[circle,white,fill=IshinDark,text centered,inner sep=1pt]{\scriptsize 市};
                \end{tikzpicture}
                \item 多様な教育法を公教育で実現
                \hspace{.2mm}
                \begin{tikzpicture}
                    \node[circle,white,fill=IshinDark,text centered,inner sep=1pt]{\scriptsize 市};
                \end{tikzpicture}
                \item 学力向上のための学校内授業に関しての民間との提携
                \hspace{.2mm}
                \begin{tikzpicture}
                    \node[circle,white,fill=IshinDark,text centered,inner sep=1pt]{\scriptsize 市};
                \end{tikzpicture}
                \item 放課後授業の充実
            \end{itemize}
        \end{small}
    \end{frame}

    \begin{frame}{府市一体の成長戦略}{首都圏に対抗できると副首都圏の確立へ}
        \begin{small}
            \begin{itemize}
                \setlength{\itemsep}{1.5mm}
                \item 二重行政の撤廃
                \item 府市連携の加速
                \item 東西二極の一極を担う経済都市圏の確立へ向けたビジョンの作成
                \item 副首都確立のための法制度の整備や政府指針の策定要望
                \item 大阪メトロの上場に向けた準備
                \item 府内市町村間における合併を含めた連携促進のための仕組みの創出
                \item \alert{府内市町村間における効果的で効率的な共同事務処理、共同調達の促進}
                \item 京阪神連携会議の創設
                \item スマートシティの実現
                \item 国際金融都市の実現へ
            \end{itemize}
        \end{small}
    \end{frame}

    \begin{frame}{府市一体の成長戦略}{新たな産業が次々と生まれ挑戦と成長が続く大阪へ}
        \begin{small}
            \begin{itemize}
                \setlength{\itemsep}{5mm}
                \item 会社設立数日本一を目指すための設立無償化・融資制度の構築
                \hspace{.2mm}
                \begin{tikzpicture}
                    \node[circle,white,fill=IshinDark,text centered,inner sep=1pt]{\scriptsize 市};
                \end{tikzpicture}
                \item ベンチャーキャピタルや海外投資家の大阪への投資を促進するための制度構築
                \item 賑わい創出にむけた公園への徹底投資
            \end{itemize}
        \end{small}
    \end{frame}

    \begin{frame}{大阪・関西万博の成功へ}{大阪の成長の起爆剤となるような大阪関西万博を}
        \begin{small}
            \begin{itemize}
                \setlength{\itemsep}{5mm}
                \item ソフトレガシーを府政・市政へ
                \begin{itemize}
                    \setlength{\itemsep}{2mm}
                    \item 万博における実証実験事業を引き継いだ実証実験促進部署の創設
                    \item 万博時に利用した無人バス等の新型交通を大阪に実装するための部署の創設
                    \item 万博時における実証実験を活かし、新型交通、オンデマンドバス、既存の交通網、大阪メトロを核にした大阪版 MAAS への準備検討                    
                \end{itemize}
                \item ハードレガシーを府政・市政へ
                \begin{itemize}
                    \setlength{\itemsep}{2mm}
                    \item 大阪館を活用した夢洲街づくりを
                \end{itemize}
            \end{itemize}
        \end{small}
    \end{frame}

    \begin{frame}{大阪・関西万博の成功へ}{大阪の成長の起爆剤となるような大阪関西万博を}
        \begin{small}
            \begin{itemize}
                \setlength{\itemsep}{2mm}
                \item 関西・大阪の企業の参加体制の構築
                \item ふるさと納税を活用したチケット販促
                \item 企業版ふるさと納税を利用した各種機運上昇策
                \item 万博跡地と統合型リゾートの相乗効果による一大観光拠点の創設
                \item 府下全域での万博プロモーション活動
                \item 万博プロモーションのための機に食・アート等の大規模イベントの開催
                \item SDGS 先進都市の実現
                \item EV の促進等ゼロカーボンを目指すための推進策の実現
            \end{itemize}
        \end{small}
    \end{frame}

\section{副首都圏の確立}
    \begin{frame}[plain,noframenumbering]
        \sectionpage
    \end{frame}

    \begin{frame}{副首都圏の確立}{}
        \begin{small}
            \begin{itemize}
                \setlength{\itemsep}{2mm}
                \item 日本の成長を牽引する「東西二極の一極」を担う「副首都圏」の確立を目指し、広 域都市インフラの充実や特区制度の活用など、ハ-ド・ソフト両面における必要な機能面を強化し、「新たな大都市制度」への大改革など、制度面の取組みを強力に推し進め、必要な基盤を整えていきます。
                \item その過程で、都市力の強化を図ることにより、産業経済の好循環を生み出し、雇用拡大・所得アップ・生産性向上を図り、福祉や教育などにおける更なる住民サ-ビスを拡充し、府民の皆様が健康で長寿で豊かな都市生活を送ることのできる大阪を実現していきます。
            \end{itemize}
        \end{small}
    \end{frame}

    \begin{frame}{第二の成長エンジンとして日本経済を牽引}{ビジョン作成への推進}
        \begin{itemize}
            \setlength{\itemsep}{5mm}
            \item 東西二極の一極を担う経済都市圏の確立へ
            \vspace{2mm}
            \begin{itemize}
                \setlength{\itemsep}{2mm}
                \item 西日本の経済成長を加速させるために、京阪神間での連携を強化。都道府県の枠を超えた副首都圏としてのビジョンを作成。
            \end{itemize}
        \end{itemize}
    \end{frame}

    \begin{frame}{第二の成長エンジンとして日本経済を牽引}{大阪・兵庫・京都による副首都圏クラスター形成}
        \begin{small}
            \begin{itemize}
                \setlength{\itemsep}{5mm}
                \item 京阪神経済圏確立のための連絡会議の設立
                \item 阪神港としての連携強化
                \item 関西国際空港・伊丹空港・神戸空港の連携強化
                \item リニア中央新幹線・北陸新幹線の新大阪駅までの早期全線開業への活動
            \end{itemize}
        \end{small}
    \end{frame}

    \begin{frame}{第二の成長エンジンとして日本経済を牽引}{府内における行政機能の最適化( Greater One Osaka )}
        \begin{small}
            \begin{itemize}
                \setlength{\itemsep}{2mm}
                \item \alert{府内市町村間における合併を含めた連携促進のための仕組みの創出 ・府内市町村間における効果的で効率的な共同事務実施の促進}
                \item \alert{府内ごみ焼却事業の広域化}
                \item \alert{大阪市消防局を中心とした府域消防の一元化(司令機能の統合)}
                \item \alert{One 大阪で取り組む府域全体のスマートシティ化}
                \item 府域一水道の実現
                \item 府市病院機構の統合
            \end{itemize}
        \end{small}
    \end{frame}

\section{成長戦略}
    \begin{frame}[plain,noframenumbering]
        \sectionpage
    \end{frame}

    \begin{frame}{成長戦略}{}
        \begin{small}
            \begin{itemize}
                \setlength{\itemsep}{2mm}
                \item 経済的副首都を目指していく大きなビジョンのもと、大阪に産業構築、それを支えるインフラ整備を進めていくことにより、大阪の経済規模を大きくしていくことを目指していきます。
            \end{itemize}
        \end{small}
    \end{frame}

    \begin{frame}{\normalsize 経済規模の拡大を目指した「ヒト」の呼び込みと産業構築}{人口対策・大阪への集客}
        \begin{small}
            \begin{itemize}
                \setlength{\itemsep}{2mm}
                \item 大阪府内人口を増やすための数値目標の策定
                \vspace{2mm}
                \begin{itemize}
                    \item 各市町村の将来人口推計を分析し、人口増に必要な方法論を検討。目標数値を設定することで、創意工夫を促し、人口増を目指す。
                \end{itemize}
                \item 御堂筋・大阪城・ベイエリア等観光拠点の活用
                \item 淀川左岸線上部のにぎわい創出
            \end{itemize}
        \end{small}
    \end{frame}

    \begin{frame}{\normalsize 経済規模の拡大を目指した「ヒト」の呼び込みと産業構築}{産業振興}
        \begin{small}
            \begin{itemize}
                \setlength{\itemsep}{2mm}
                \item アフターコロナに向けた産業支援
                \item 成長分野のベンチャー支援
                \item 対内投資促進
                \item \alert{商店街の活性化に向けた支援策の強化・海外進出企業の支援}
                \item 産業局のブランディング、支援内容の強化
                \item 挑戦を支えるセーフティネット
                \item シェアリングエコノミーの推進体制の強化
                \item シルバー人材を活用するための労働市場の整備
                \item 空家の再生による活気創造
                \item 企業の競争力強化のための税制度の見直し
            \end{itemize}
        \end{small}
    \end{frame}

    \begin{frame}{経済成長を加速させるインフラ整備・国際化}{インフラ整備}
        \begin{small}
            \begin{itemize}
                \setlength{\itemsep}{2mm}
                \item 御堂筋のフルモール化へ
                \item 都市再生緊急整備地域指定による新大阪の再整備
                \item 大阪城東部地域の開発
                \item うめきた 2 期開発
                \item 未利用施設の再生
                \item 天王寺動物園の大幅な魅力向上
                \item 大中規模公園の民間活力の導入
                \item 淀川左岸線上部のサイクルロード
                \item \alert{EV 充電設備の設置補助}
            \end{itemize}
        \end{small}
    \end{frame}

    \begin{frame}{経済成長を加速させるインフラ整備・国際化}{大阪関西万博・統合型リゾート}
        \begin{small}
            \begin{itemize}
                \setlength{\itemsep}{2mm}
                \item 関西・大阪の企業の参加体制の構築
                \item ふるさと納税を活用したチケット販促
                \item 万博跡地と統合型リゾートの相乗効果による観光拠点の創設
                \item  依存症対策センターの設置の前倒し
                \item \alert{ゲーム・ネット・スマホ・ギャンブル等も含めた総合的な依存症対策}
            \end{itemize}
        \end{small}
    \end{frame}

\section{改革}
    \begin{frame}[plain,noframenumbering]
        \sectionpage
    \end{frame}

    \begin{frame}{改革}{}
        \begin{small}
            \begin{itemize}
                \item 市民サービスの最大化を目指し、効果的かつ効率的な行政を実現していきます。そのための府政改革・市政改革を進めていくための体制を構築し、民間と同等のレベルでサービスの向上を行うための、採用・人事制度を整えます。その上で民間と比較して周回遅れとなっているデジタル化を進めるために、必要な人材の確保し、DX 化を推し進めていきます。
            \end{itemize}
        \end{small}
    \end{frame}

    \begin{frame}{時代に合わせた業務レベルの引き上げ}{公務員改革}
        \begin{small}
            \begin{itemize}
                \setlength{\itemsep}{2mm}
                \item 引き続き所属長ポスト等について広く内外から公募
                \item 市民サービスの最大化のための、リモートワーク等の体制整備
                \item 外郭団体の削減の継続
                \item 民間との人事交流
                \item 全職員の ICT スキルやリテラシー強化
                \item 公務員のリスキリング制度の構築
            \end{itemize}
        \end{small}
    \end{frame}

    \begin{frame}{時代に合わせた業務レベルの引き上げ}{財政改革}
        \begin{small}
            \begin{itemize}
                \setlength{\itemsep}{2mm}
                \item \alert{未利用地の売却、利活用の促進}
                \item 引き続き団体補助ではなく、事業補助を基本とする方針維持
                \item 新公会計制度に基づく、フルコスト計算による事業判断制度の構築
                \item 未収金対策の促進
                \item 公共施設のカルテ化により適正計画の作成、維持管理コストの最小化の実現
            \end{itemize}
        \end{small}
    \end{frame}
    
    \begin{frame}{不断の改革とデジタル技術の積極導入}{府政市政改革}
        \begin{small}
            \begin{itemize}
                \setlength{\itemsep}{2mm}
                \item 市政改革プランの抜本的なリニューアル
                \item ごみ収集業務の民間委託の推進
                \item 幼稚園・保育園の民営化の推進
                \item 住宅供給公社の経営形態の見直し
                \item 大阪消防庁の設立
                \item 府市病院機構の統合
            \end{itemize}
        \end{small}
    \end{frame}

    \begin{frame}{不断の改革とデジタル技術の積極導入}{議会改革}
        \begin{small}
            \begin{itemize}
                \setlength{\itemsep}{5mm}
                \item 議員定数の削減を進める
                \begin{itemize}
                    \setlength{\itemsep}{2mm}
                    \item \alert{人口推計を睨みながらの議員定数の削減}
                    \item \alert{大阪市議会の過半数の議席獲得後の議員定数削減}
                \end{itemize}
                \item 議会のオンライン化等の ICT 推進
                \begin{itemize}
                    \setlength{\itemsep}{2mm}
                    \item 民間レベルのデジタル技術を積極導入し、議会の生産性向上にチャレンジ
                \end{itemize}
            \end{itemize}
        \end{small}
    \end{frame}

    \begin{frame}{不断の改革とデジタル技術の積極導入}{DX}
        \begin{small}
            \begin{itemize}
                \setlength{\itemsep}{2mm}
                \item 府市の事業において、自動化できる業務をデジタル化していくため、必要な人材の確保、全部署に対してプロセス把握の専門家が介入できる体制整備
                \item マイナンバーを核にし、デジタルでできるサービスをすべてデジタル化
                \item 行政プロセスに踏み込んだ抜本的な BPR の実施
                \item デジタル統括室への IT 人材の積極採用
                \hspace{.2mm}
                \begin{tikzpicture}
                    \node[circle,white,fill=IshinDark,text centered,inner sep=1pt]{\scriptsize 市};
                \end{tikzpicture}
            \end{itemize}
        \end{small}
    \end{frame}

\section{子育て教育サービス}
    \begin{frame}[plain,noframenumbering]
        \sectionpage
    \end{frame}

    \begin{frame}{子育て教育サービス}{}
        \begin{small}
            \begin{itemize}
                \setlength{\itemsep}{2mm}
                \item 維新版教育無償化モデルの実現も含め、日本一子どもを育てやすい都市を目指し、各種サービス・支援体制を整備していきます。                
            \end{itemize}
        \end{small}
    \end{frame}

    \begin{frame}{子育て支援}{新規施策}
        \begin{small}
            \begin{itemize}
                \setlength{\itemsep}{2mm}
                \item 貧困対策として養育費の確保施策の推進
                \item 派遣型の病児・病後児保育の推進
                \item 新婚家庭の大阪への誘導
                \item 放課後議業の充実
                \item 公立保育所における無償のおむつ補助
                \hspace{.2mm}
                \begin{tikzpicture}
                    \node[circle,white,fill=IshinDark,text centered,inner sep=1pt]{\scriptsize 市};
                \end{tikzpicture}
                \item インフルエンザワクチンの子どもへの助成
                \hspace{.2mm}
                \begin{tikzpicture}
                    \node[circle,white,fill=IshinDark,text centered,inner sep=1pt]{\scriptsize 市};
                \end{tikzpicture}
            \end{itemize}
        \end{small}
    \end{frame}

    \begin{frame}{子育て支援}{継続施策}
        \begin{small}
            \begin{itemize}
                \setlength{\itemsep}{2mm}
                \item 待機児童対策
                \item 子供医療費助成の維持
                \item 児童虐待防止体制の連携強化
                \item 里親の推進
                \item 妊婦健康診査への公費負担の維持
                \hspace{.2mm}
                \begin{tikzpicture}
                    \node[circle,white,fill=IshinDark,text centered,inner sep=1pt]{\scriptsize 市};
                \end{tikzpicture}
            \end{itemize}
        \end{small}
    \end{frame}

    \begin{frame}{教育}{新規施策}
        \begin{small}
            \begin{itemize}
                \setlength{\itemsep}{2mm}
                \item 府立高校の公設民営化への推進
                \hspace{.2mm}
                \begin{tikzpicture}
                    \node[circle,white,fill=IshinMain,text centered,inner sep=1pt]{\scriptsize 府};
                \end{tikzpicture}
                \item 動画授業の開発、全生徒への展開
                \hspace{.2mm}
                \begin{tikzpicture}
                    \node[circle,white,fill=IshinDark,text centered,inner sep=1pt]{\scriptsize 市};
                \end{tikzpicture}
                \item 学校への学習補助要員の増員・多様化
                \hspace{.2mm}
                \begin{tikzpicture}
                    \node[circle,white,fill=IshinDark,text centered,inner sep=1pt]{\scriptsize 市};
                \end{tikzpicture}
                \item 既存の教育と一線を画す、生きる力重視の不登校特例校
                \hspace{.2mm}
                \begin{tikzpicture}
                    \node[circle,white,fill=IshinDark,text centered,inner sep=1pt]{\scriptsize 市};
                \end{tikzpicture}
                \item 教育センターの強化
                \hspace{.2mm}
                \begin{tikzpicture}
                    \node[circle,white,fill=IshinDark,text centered,inner sep=1pt]{\scriptsize 市};
                \end{tikzpicture}
                \item \alert{校務プリントの電子化}
                \hspace{.2mm}
                \begin{tikzpicture}
                    \node[circle,white,fill=IshinDark,text centered,inner sep=1pt]{\scriptsize 市};
                \end{tikzpicture}
                \item \alert{不登校特例校の設置に向けた市町村に対する指導助言}
                \item スクールカウンセラー、スクールソーシャルワーカーの充実
            \end{itemize}
        \end{small}
    \end{frame}

    \begin{frame}{教育}{継続施策}
        \begin{small}
            \begin{itemize}
                \setlength{\itemsep}{2mm}
                \item いじめ対策
                \item 校長公募
                \item 校務分掌の見直し
                \item 教育現場におけるリスクマネジメントの強化
                \item 学力向上
                \item 英語教育の推進
                \item プログラミング教育
            \end{itemize}
        \end{small}
    \end{frame}

    \begin{frame}{教育}{継続施策}
        \begin{small}
            \begin{itemize}
                \setlength{\itemsep}{2mm}
                \item 学校図書の強化
                \item 教科書採択の最適化
                \item 地域ボランティアの導入
                \item 部活動の在り方改革
                \item 学校適正配置
                \item 児童急増・校地狭隘校において環境改善
                \item 学校跡地活用
            \end{itemize}
        \end{small}
    \end{frame}

\section{市民サービス}
    \begin{frame}[plain,noframenumbering]
        \sectionpage
    \end{frame}
    
    \begin{frame}{市民サービス}{}
        \begin{small}
            \begin{itemize}
                \setlength{\itemsep}{2mm}
                \item 様々な改革・成長戦略により、持続的な住民サービスの拡充が可能になってきています。大阪にこれから住んでみたい、ずっと住みたいと思ってもらえるような住民サービスを実現し、人が集いとどまる大阪に向けて邁進していきます。
            \end{itemize}
        \end{small}
    \end{frame}

    \begin{frame}{保健医療・福祉}{新規施策}
        \begin{small}
            \begin{itemize}
                \setlength{\itemsep}{2mm}
                \item インフルエンザワクチンの子どもへの助成
                \hspace{.2mm}
                \begin{tikzpicture}
                    \node[circle,white,fill=IshinDark,text centered,inner sep=1pt]{\scriptsize 市};
                \end{tikzpicture}
                \item \alert{ゲーム・ネット・スマホ・ギャンブル等も含めた総合的な依存症対策}
                \item 買い物難民対策
                \item 特定検診の拡充・充実
                \item ヤングケアラー対策の充実
                \item ダブルケア対策の充実
                \item 万博に向けてのバリアフリー化、ユニバーサルデザインを考慮したインフラ整備
            \end{itemize}
        \end{small}
    \end{frame}

    \begin{frame}{保健医療・福祉}{継続施策}
        \begin{small}
            \begin{itemize}
                \setlength{\itemsep}{2mm}
                \item がん検診・特定検診の受診率向上
                \item 各種ワクチン補助体制の拡充
                \item 特別養護老人ホームの整備等
                \item 生活保護の適正化
                \item 発達障がい者支援の充実
                \item 重症心身障がい児者支援のためのショートステイの拡充
                \item 公共施設・公共交通・道路のバリアフリー化
                \item 手話に関する施策の充実
            \end{itemize}
        \end{small}
    \end{frame}

    \begin{frame}{住民生活}{新規施策}
        \begin{small}
            \begin{itemize}
                \setlength{\itemsep}{2mm}
                \item 地域活動への新たな交付金等支援策の導入
                \item 地域活動の委託受付
                \item 区役所等役所窓口のオンライン化
                \item 図書館のオンライン化、宅配サービス
                \item 公園への民間活力による遊具・売店等の投資
                \item \alert{河川やお堀等の水質改善}
                \item \alert{大阪市内全面禁煙に向けての喫煙所整備}
            \end{itemize}
        \end{small}
    \end{frame}

    \begin{frame}{住民生活}{継続施策}
        \begin{small}
            \begin{itemize}
                \setlength{\itemsep}{2mm}
                \item 放置自転車対策
                \item 自転車通行空間のネットワーク化
                \item キャッシュレス決裁の促進
                \item 全庁横断的な空家対策
                \item ごみ屋敷も含めた老朽危険家屋対策
                \item \alert{府域における公衆浴場活性化に向けた取り組みの推進}
                \item マイナンバー制度の活用
                \item 防犯カメラの運用
                \item 犬・猫の理由なき殺処分0
            \end{itemize}
        \end{small}
    \end{frame}

    \begin{frame}{安心安全・防災}{新規施策}
        \begin{small}
            \begin{itemize}
                \setlength{\itemsep}{5mm}
                \item 電柱の地中化の加速
                \item 複数事業目的にまたがる高齢者全戸訪問
                \item 地域活動協議会を活用した防災対策
            \end{itemize}
        \end{small}
    \end{frame}

    \begin{frame}{安心安全・防災}{継続施策}
        \begin{small}
            \begin{itemize}
                \setlength{\itemsep}{2mm}
                \item 大規模災害対策の推進
                \item 避難所としての公共施設のインフラ強化
                \item ICT を利用した災害時の情報収集と発信
                \item 民間施設の防災時の活用強化
                \item 大阪市内各区の防災強化
                \item 消防訓練の充実
            \end{itemize}
        \end{small}
    \end{frame}

    \begin{frame}{安心安全・防災}{継続施策}
        \begin{small}
            \begin{itemize}
                \setlength{\itemsep}{2mm}
                \item 地域防災組織の強化
                \item 地域掲示板の補助、デジタルサイネージ等を活用した多機能化
                \item 密集市街地整備と住宅等の耐震化
                \item インバウンドも含めた帰宅困難者対策・安全対策
                \item 猛暑災害対策の推進
            \end{itemize}
        \end{small}
    \end{frame}
\end{document}